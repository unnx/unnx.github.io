\documentclass{article}
\usepackage[utf8]{inputenc}
\usepackage[russian]{babel}
\usepackage{cmap}
\usepackage{amsfonts,amsmath}
\usepackage{geometry}
\usepackage{fixint}
\usepackage{rumathgrk1}
\geometry{verbose,a4paper,tmargin=1cm,bmargin=1.5cm,lmargin=0.5cm,rmargin=0.5cm}
\pdfcompresslevel=9

\begin{document}
	
	\textbf{Энергия электромагнитных волн}\\

	Электромагнитные волны переносят энергию. Известно, что плотность потока энергии можно получить, умножив плотность энергии на скорость волны (см. энергия упругой волны). Рассмотрим распространение электромагнитной волны в вакууме. Плотость энергии электрического поля и плотность энергии магнитного поля в сумме дают плотность полной энергии:
	\begin{equation}
		w=w_E+w_H=\frac{\varepsilon_0E^2}{2}+\frac{\mu_0H^2}{2}
	\end{equation}

	В данной точке пространства векторы $\vec E$ и $\vec H$ изменяются в одинаковой фазе (это справедливо только для непроводящей среды). Поэтому соотношение между амплитудными значениями $E_m\sqrt{\varepsilon\varepsilon_0}=H_m\sqrt{\mu\mu_0}$ (см. плоская электромагнитная волна) верно и для их мгновенных значений:
	\begin{equation}
		E_m\sqrt{\varepsilon_0}=H\sqrt{\mu_0}
	\end{equation}

	Отсюда следует, что плотности энергии электрического и магнитного полей волны в каждый момент времени одинаковы: $w_E=w_H$. Выражению для плотности полной энергии можно придать вид:
	\begin{equation}
		w=\frac{1}{2}(E\sqrt{\varepsilon_0})(E\sqrt{\varepsilon_0})+\frac{1}{2}(H\sqrt{\mu_0})(H\sqrt{\mu_0})=\sqrt{\varepsilon_0\mu_0}EH=\frac{1}{c}EH
	\end{equation}

	Умножив это выражение на скорость волны, получим модуль плотности потока энергии:
	\begin{equation}
		S = wc = EH
	\end{equation}

	Векторы $\vec E$ и $\vec H$ взаимно перпендикулярны и образуют с направлением распространения волны правовинтовую систему, поэтому направление $[\vec E;\vec H]$ совпадает с направлением переноса энергии, а его модуль равен плотности потока энергии. Этот вектор называется вектором Пойнтинга:
	\begin{equation}
		\vec S = [\vec E;\vec H]
	\end{equation}

	Поток $\Phi$ энергии электромагнитной волны через некоторую поверхность $F$ можно найти так:
	\begin{equation}
		\Phi = \int_F \vec Sd\vec F
	\end{equation}

\end{document}
