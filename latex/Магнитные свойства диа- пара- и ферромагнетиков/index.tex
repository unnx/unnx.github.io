\documentclass{article}
\usepackage[utf8]{inputenc}
\usepackage[russian]{babel}
\usepackage{cmap}
\usepackage{amsfonts,amsmath}
\usepackage{geometry}
\usepackage{fixint}
\usepackage{rumathgrk1}
\geometry{verbose,a4paper,tmargin=1cm,bmargin=1.5cm,lmargin=0.5cm,rmargin=0.5cm}
\pdfcompresslevel=9

\begin{document}
	
	\textbf{Магнитные свойства диа- пара- и ферромагнетиков}\\

	Вещества условно разделяются на диамагнетики, парамагнетики и ферромагнетики. Рассмотрим различия между их магнитными свойствами.\\

	Под действием внешнего магнитного поля в диамагнетиках появляется индуцированный магнитный момент, направленный против поля. Магнитная проницаемость диамагнетиков чуть меньше единицы:
	\begin{equation}
		\mu \leq 1
	\end{equation}

	Диамагнетизм обнаруживается только у тех веществ, у атомов которых нет собственного магнитного момента.\\

	Если же атомы имеют собственный магнитный момент, вещество является парамагнитным. Магнитное поле стремится установить моменты по полю. Таким образом, парамагнетик усиливает внешнее поле. Магнитная проницаемость парамагнетиков чуть больше единицы:
	\begin{equation}
		\mu \geq 1
	\end{equation}

	Особый класс веществ образуют ферромагнетики -- вещества, обладающие спонтанной намагниченностью в отсутствие внешнего магнитного поля. Намагниченность ферромагнетиков зависит от напряженности внешнего поля сложным образом. Обычно эту зависимость представляют как зависимость $B$ от $H$. Для этой зависимости характерен гистерезис. Магнитная проницаемость ферромагнетиков много больше единицы:
	\begin{equation}
		\mu >> 1
	\end{equation}

\end{document}
