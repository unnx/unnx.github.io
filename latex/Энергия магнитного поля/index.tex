\documentclass{article}
\usepackage[utf8]{inputenc}
\usepackage[russian]{babel}
\usepackage{cmap}
\usepackage{amsfonts,amsmath}
\usepackage{geometry}
\usepackage{fixint}
\usepackage{rumathgrk1}
\geometry{verbose,a4paper,tmargin=1cm,bmargin=1.5cm,lmargin=0.5cm,rmargin=0.5cm}
\pdfcompresslevel=9

\begin{document}
	
	\textbf{Энергия магнитного поля}\\

	Представим соленоид с параллельно подключенным сопротивлением $R$ и источником тока $I$. Если отключить источник, появляется ЭДС самоиндукции. Работа $dA$, совершаемая магнитным полем за $dt$ равна:
	\begin{equation}
		dA = \mathcal{E}_s I dt = -\frac{d\Psi}{dt}Idt = -Id\Psi
	\end{equation}

	Если индуктивность не зависит от $I$, то $d\Psi=LdI$ 
	\begin{equation}
		dA = -LIdI
	\end{equation}
	\begin{equation}
		A = -\int_I^0 LI dI = \frac{LI^2}{2}
	\end{equation}

	Работа, создаваемая полем, равна запасу его потенциальной энергии. Отсюда:
	\begin{equation}
		W = \frac{LI^2}{2}
	\end{equation}

	В случае бесконечного соленоида:
	\begin{equation}
		L = \mu_0\mu n^2 V,\;\;\;H=nI
	\end{equation}

	Отсюда:
	\begin{equation}
		W = \frac{\mu_0\mu H^2}{2}V
	\end{equation}

	Плотность энергии, соответственно, равна:
	\begin{equation}
		w = \frac{\mu_0\mu H^2}{2}
	\end{equation}

	Или
	\begin{equation}
		w = \frac{HB}{2}
	\end{equation}

	Внутри изотропных магнетиков $\vec B || \vec H$, поэтому:
	\begin{equation}
		w = \frac{(\vec H;\vec B)}{2}
	\end{equation}
\end{document}
