\documentclass{article}
\usepackage[utf8]{inputenc}
\usepackage[russian]{babel}
\usepackage{cmap}
\usepackage{amsfonts,amsmath}
\usepackage{geometry}
\usepackage{fixint}
\usepackage{rumathgrk1}
\geometry{verbose,a4paper,tmargin=1cm,bmargin=1.5cm,lmargin=0.5cm,rmargin=0.5cm}
\pdfcompresslevel=9

\begin{document}
	
	\textbf{Вектор электрического смещения}\\

	Из теоремы Гаусса для вектора $\vec E$ и теоремы Остроградского-Гаусса следует:
	\begin{equation}
		(\vec\nabla;\vec E) = \frac{q}{V\varepsilon_0}
	\end{equation}

	Известно, что суммарная плотность зарядов состоит из плотности связанных и сторонних зарядов ($q/V = \rho + \rho'$), поэтому:
	\begin{equation}
		(\vec\nabla;\vec E) = \frac{1}{\varepsilon_0}(\rho+\rho')
	\end{equation}

	Известно также, что плотность связанных зарядов выражается через вектор поляризованности:
	\begin{equation}
		(\vec\nabla;\vec E) = \frac{1}{\varepsilon_0}(\rho-(\vec\nabla;\vec P))
	\end{equation}

	Отсюда следует, что
	\begin{equation}
		(\vec\nabla;\varepsilon_0\vec E + \vec P) = \rho
	\end{equation}

	Величина $\varepsilon_0\vec E + \vec P$ называется вектором электрического смещения $\vec D$:
	\begin{equation}
		\vec D = \varepsilon_0\vec E + \vec P
	\end{equation}

	Распишем векторе $\vec P$:
	\begin{equation}
		\vec D = \varepsilon_0\vec E + \kappa\varepsilon_0\vec E = \varepsilon_0(1+\kappa)\vec E
	\end{equation}

	Безразмерная величина $1+\kappa$ называется диэлектрической проницаемостью среды, и обозначается через $\varepsilon$. Таким образом:
	\begin{equation}
		\vec D = \varepsilon_0\varepsilon\vec E
	\end{equation}

	Смысл вектора $\vec D$ в том, что зная его можно сразу найти поле, не рассчитывая влияние поля поверхностно-связанных зарядов $\vec E'$ на $\vec E_0$. Найти $\vec D$ можно по соотношению:
	\begin{equation}
		(\vec\nabla;\vec D) = \rho
	\end{equation}

	Или по другому соотношению, вытекающему из предыдущего:
	\begin{equation}
		\oint_S \vec D d\vec S = q
	\end{equation}

	Выходит, что $\vec D$ связан только со сторонними зарядами.
\end{document}
