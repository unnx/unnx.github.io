\documentclass{article}
\usepackage[utf8]{inputenc}
\usepackage[russian]{babel}
\usepackage{cmap}
\usepackage{amsfonts,amsmath}
\usepackage{geometry}
\usepackage{fixint}
\usepackage{rumathgrk1}
\geometry{verbose,a4paper,tmargin=1cm,bmargin=1.5cm,lmargin=0.5cm,rmargin=0.5cm}
\pdfcompresslevel=9

\begin{document}
	
	\textbf{Сила Ампера и сила Лоренца}\\

	Экспериментально установлено, что сила, действующая на заряд $q$, движущийся со скоростью $\vec v$ со стороны магнитного поля индукции $\vec B$ равна:
	\begin{equation}
		\vec F = q[\vec v;\vec B]
	\end{equation}

	Помимо магнитных сил, на заряд действуют и электростатические, поэтому полная сила равна:
	\begin{equation}
		\vec F = q\vec E + q[\vec v;\vec B]
	\end{equation}

	Эта сила называется силой Лоренца.\\

	Найдем силу $d\vec F$, действующую со стороны поля на элемент проводника $d\vec l$. На каждый из носителей тока в проводнике действует сила:
	\begin{equation}
		\vec F = e[\vec u;\vec B]
	\end{equation}

	В элементе провода длины $dl$ содержится $nSdl$ зарядов $e$, поэтому сила, действующая на этот элемент:
	\begin{equation}
		d\vec F = [ne\vec u;\vec B]Sdl
	\end{equation} 

	$ne\vec v$ есть плотность тока $\vec j$, поэтому
	\begin{equation}
		d\vec F = [\vec j;\vec B]Sdl
	\end{equation}

	$\vec jSdl$ можно заменить на $jSd\vec l$, если $\vec j \uparrow\uparrow \vec l$:
	\begin{equation}
		d\vec F = I[d\vec l;\vec B]
	\end{equation}

	Эта сила называется силой Ампера.

\end{document}
