\documentclass{article}
\usepackage[utf8]{inputenc}
\usepackage[russian]{babel}
\usepackage{cmap}
\usepackage{amsfonts,amsmath}
\usepackage{geometry}
\usepackage{fixint}
\usepackage{rumathgrk1}
\geometry{verbose,a4paper,tmargin=1cm,bmargin=1.5cm,lmargin=0.5cm,rmargin=0.5cm}
\pdfcompresslevel=9

\begin{document}
	
	\textbf{Наличие свободных электронов в металлах}\\

	Металлы -- хорошие проводники. Требовалось выяснить природу носителей зарядов в них.\\

	В 1901 году Рикке поставил опыт по определению этой природы. Он взял два медных цилиндра и один алюминиевый, с тщательно отшлифованными торцами. Соединив их в последовательности "медь-алюминий-медь", он начал пропускать через полученный проводник заряд. Спустя год цилиндры были исследованы: они не изменились. Следовательно, заряд переносят не атомы металла, а другие частицы, например, электроны. \\

	Чтобы доказать причастность электронов к переносу заряда в металлах, Лоренц предложил другой опыт. Если в металлическом проводнике есть свободные заряженные частицы, при торможении проводника они должны некоторое время продолжать двигаться по инерции. В результате будет перенесен заряд. Пусть проводник движется со скоростью $\vec v_0$. Начнем тормозить его с ускорением $\vec a$. Такое же ускорение можно сообщить носителям в неподвижном проводнике, если создать в нем электрическое поле напряженности $\vec E = -\frac{m\vec a}{e'}$, т.е. приложить к концам провода разность потенциалов:
	\begin{equation}
		\varphi_1 - \varphi_2 = \int_1^2 \vec E d\vec l = -\int_1^2\frac{m\vec a}{e'} d\vec l = -\frac{mal}{e'}
	\end{equation}
	В этом случае по проводу течет ток с силой
	\begin{equation}
		I = (\varphi_1-\varphi_2)/R
	\end{equation}
	Поэтому за время $dt$ по проводу пройдет заряд
	\begin{equation}
		dq = Idt = -\frac{mal}{e'R}dt = -\frac{ml}{e'R}dv
	\end{equation}
	Заряд, протекающий при торможении, равен:
	\begin{equation}
		q = \int dq = -\int_{v_0}^0 \frac{ml}{e'R}dv = \frac{mlv_0}{e'R}
	\end{equation}

	Опыт Лоренца был поставлен Толменом и Стюартом в 1916 г. Вычисленный удельный заряд был близок к удельному заряду электрона, что доказало их причастность к переносу зарядов в металлах.
\end{document}
