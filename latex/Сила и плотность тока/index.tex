\documentclass{article}
\usepackage[utf8]{inputenc}
\usepackage[russian]{babel}
\usepackage{cmap}
\usepackage{amsfonts,amsmath}
\usepackage{geometry}
\usepackage{fixint}
\usepackage{rumathgrk1}
\geometry{verbose,a4paper,tmargin=1cm,bmargin=1.5cm,lmargin=0.5cm,rmargin=0.5cm}
\pdfcompresslevel=9

\begin{document}
	
	\textbf{Сила и плотность тока}\\

	Если в проводнике носители заряда движутся с некоторой средней скоростью $<\vec u>$, говорят, что в нем течет ток. Характеристикой тока является его сила $I$:
	\begin{equation}
		I = \frac{dq}{dt}
	\end{equation}

	$dq$ -- заряд, проходящий за время $dt$ сквозь поверхность, перпендикулярную движению зарядов.\\

	Сила тока зависит от площади сечения проводника. Введем характеристику $\vec j$, зависящую только от зарядов:
	\begin{equation}
		dI = (\vec j;d\vec S)
	\end{equation}
	\begin{equation}
		I = \int_S \vec j d\vec S
	\end{equation}

	Эта характеристика называется плотностью тока. Она численно равна заряду, проходящему за единицу времени через поверхность с единичной площадью.\\

	Ток, в котором $\vec j$ не меняется со временем называется постоянным. Для него справедливо:
	\begin{equation}
		I = \frac{q}{t}
	\end{equation}

	Возьмем внутри проводника замкнутую поверхность $S$, изнутри которой вытекает заряд. По закону сохранения заряда скорость убывания заряда равна заряду, пересекающему $S$ в единицу времени:
	\begin{equation}
		\frac{-dq}{dt} = \oint_S \vec j d\vec S
	\end{equation}

	Выразив заряд через его плотность, получим:
	\begin{equation}
		\oint_S \vec j d\vec S = -\frac{d}{dt}\int_V \rho dV = -\int_V \frac{\partial \rho}{\partial t}dV
	\end{equation}

	По теореме Остроградского-Гаусса:
	\begin{equation}
		\int_V (\vec\nabla;\vec j)dV = -\int_V\frac{\partial\rho}{\partial t}dV
	\end{equation}

	Отсюда:
	\begin{equation}
		(\vec\nabla;\vec j) = -\frac{\partial\rho}{\partial t}
	\end{equation}

	Это соотношение называется уравнением непрерывности.\\

	В случае стационарного тока плотность заряда не зависит от времени, поэтому $(\vec\nabla;\vec j) = 0$.
\end{document}
