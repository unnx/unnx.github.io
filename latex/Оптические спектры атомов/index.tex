\documentclass{article}
\usepackage[utf8]{inputenc}
\usepackage[russian]{babel}
\usepackage{cmap}
\usepackage{amsfonts,amsmath}
\usepackage{geometry}
\usepackage{fixint}
\usepackage{rumathgrk1}
\geometry{verbose,a4paper,tmargin=1cm,bmargin=1.5cm,lmargin=0.5cm,rmargin=0.5cm}
\pdfcompresslevel=9

\begin{document}

	\textbf{Оптические спектры атомов}\\

	Частоты спектральных линий атома водорода выражаются обобщенной формулой Бальмера:
	\begin{equation}
		\omega = R(\frac{1}{m^2}-\frac{1}{n^2})
	\end{equation}

	$m$ -- значение главного квантого числа после электрона, $n$ -- до него.\\

	Выделяют несколько серий спектральных линий водорода:\\

	Лаймана -- $np \rightarrow 1s,\;\;\;n=2,3,...$\\

	Бальмера -- $np \rightarrow 2s,\;ns\rightarrow 2p,\;nd\rightarrow 2p\;\;\;n=3,4,...$\\

	Пашена -- $nf \rightarrow 3d,\;\;\;n=4,5,...$\\

	Брэкета -- $ng \rightarrow 4f,\;\;\;n=5,6,...$\\

	Частоты спектральных линий щелочных металлов выражаются похожей формулой:
	\begin{equation}
		\omega = R(\frac{1}{(m+\alpha_{l_2})^2}-\frac{1}{(n+\alpha_{l_1})^2})
	\end{equation}

	$\alpha_l$ -- соответствующая орбитальному квантовому числу т.н. поправка Ридберга.\\

	Выделяют серии спектральных линий щелочных металлов:\\

	Резкая -- $nS \rightarrow 3P\;\;\;n=4,5,...$\\

	Главная -- $nP \rightarrow 3S\;\;\;n=3,4,...$\\

	Диффузная -- $nD \rightarrow 3P\;\;\;n=3,4,...$\\

	Основная -- $nF \rightarrow 3D\;\;\;n=4,5,...$\\

	Взаимодействие спинового и орбитального моментов электрона обладает энергией. Поэтому некоторые уровни (термы) расщепляются. Полный момент импульса $j$ электрона равен:
	\begin{equation}
		M_j = h\sqrt{j(j+1)},\;\;\;j=l+s,\;j=|l-s|
	\end{equation}

	Таким образом, в атомах водорода, где электрон один, а также в атомах щелочных металлов, где спектры определяются только переходами валентного электрона, все уровни с $l\neq s$ расщепляются на два подуровня.\\

	В многоэлектронных атомах механические моменты электронов могут складываться в результирующий момент $M_J$ двумя способами.\\

	1) LS-связь: моменты одного типа (орбитальные или спиновые) взаимодействуют сильнее с моментами своего типа, складываясь, соответственно в $M_L$ и $M_S$, которые образуют полный момент атома $M_J$. 

	2) jj-связь: пары моментов разных типов взаимодействуют сильнее, чем моменты одного типа, образуя $M_j$, которые складываются в $M_J$.\\

	При LS-связи возможные значения $J$ равны:
	\begin{equation}
		J=L+S,\;L+S-1,\;...|L-S|
	\end{equation}

	Полные магнитные моменты атома связаны с $L$, $S$ и $J$ соотношениями:
	\begin{equation}
		\mu_L = -\mu_\text{Б}\sqrt{L(L+1)}
	\end{equation} 
	\begin{equation}
		\mu_S = -2\mu_\text{Б}\sqrt{S(S+1)}
	\end{equation} 
	\begin{equation}
		\mu_J = -\mu_\text{Б}g\sqrt{J(J+1)}
	\end{equation} 
	\begin{equation}
		\mu_{Jz} = -\mu_\text{Б}gm_J\;\;\;(m_J=-J,\;-J+1,\;...,\;J-1,\;J)
	\end{equation}

	Здеcь $g$ -- т.н. фактор Ланде:
	\begin{equation}
		g=1+\frac{J(J+1)+S(S+1)-L(L+1)}{2J(J+1)}
	\end{equation}

	При помещении атомов вещества в магнитное поле наблюдается т.н. эффект Зеемана, выражающийся в расщеплении энергетических уровней.\\

	Атом в магнитном поле приобретает дополнительную энергию:
	\begin{equation}
		\Delta E = -\mu_{J_B} B = \mu_\text{Б}gBm_J\;\;\;(m_J=-J,\;-J+1,\;...,\;J-1,\;J)
	\end{equation}

	При переходе между зеемановскими подуровнями действует правило отбора:
	\begin{equation}
		\Delta m_J = 0,\;1
	\end{equation}

	Однако, при сильном магнитном поле связь между $M_L$ и $M_S$ разрывается, и они проецируются на направление поля независимо друг от друга:
	\begin{equation}
		\Delta E = \mu_\text{Б} B m_L + 2\mu_\text{Б} B m_S = \mu_\text{Б}B(m_L+2m_S)
	\end{equation}

	В этом случае для переходов имеют место правила отбора:
	\begin{equation}
		\Delta m_L=0,\;\Delta m_S = 0
	\end{equation}




	
\end{document}
