\documentclass{article}
\usepackage[utf8]{inputenc}
\usepackage[russian]{babel}
\usepackage{cmap}
\usepackage{amsfonts,amsmath}
\usepackage{geometry}
\usepackage{fixint}
\usepackage{rumathgrk1}
\geometry{verbose,a4paper,tmargin=1cm,bmargin=1.5cm,lmargin=0.5cm,rmargin=0.5cm}
\pdfcompresslevel=9

\begin{document}

	\textbf{Энергия взаимодействия системы зарядов}\\
	
	Потенциальная энергия заряда $q_1$, находящегося в поле заряда $q_2$, определяется выражением:
	\begin{equation}
		W_p = \frac{1}{4\pi\varepsilon_0}\frac{q_1q_2}{r^2}
	\end{equation}

	Можно считать $W_p$ взаимной потенциальной энергией зарядов $q_1$ и $q_2$. 

	Энергия взаимодействия системы зарядов определяется как полусумма энергий взаимодействия пар:
	\begin{equation}
		W_{p} = \frac{1}{2}\sum_{i\neq k}W_{pik}(r_{ik})
	\end{equation}

	При этом:
	\begin{equation}
		W_{pik} = \frac{1}{4\pi\varepsilon_0}\frac{q_iq_k}{r_{ik}}
	\end{equation}

	Подставим:
	\begin{equation}
		W_p = \frac{1}{2}\sum_{i\neq k}\frac{1}{4\pi\varepsilon_0}\frac{q_iq_k}{r_{ik}}
	\end{equation}

	Вынесем из-под знака суммы исследуемые заряды, оставив там заряды, "создающие поле".
	\begin{equation}
		W_p = \frac{1}{2}\sum_{i=1}^nq_i\sum_{k=1\neq i}^n\frac{1}{4\pi\varepsilon_0}\frac{q_k}{r_{ik}}
	\end{equation}

	Ясно, что $\sum_{k=1\neq i}^n\frac{1}{4\pi\varepsilon_0}$ определяет суммарный потенциал $\varphi_i$, создаваемый всеми зарядами, кроме $q_i$ в той точке, где находится $q_i$. Таким образом, выражение для потенциальной энергии взаимодействия можно переписать:
	\begin{equation}
		W_p = \frac{1}{2}\sum_{i=1}^n q_i\varphi_i
	\end{equation}

\end{document}
