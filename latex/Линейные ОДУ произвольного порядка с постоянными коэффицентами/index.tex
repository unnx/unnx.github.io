\documentclass{article}
\usepackage[utf8]{inputenc}
\usepackage[russian]{babel}
\usepackage{cmap}
\usepackage{amsfonts,amsmath}
\usepackage{geometry}
\usepackage{fixint}
\usepackage{rumathgrk1}
\geometry{verbose,a4paper,tmargin=1cm,bmargin=1.5cm,lmargin=0.5cm,rmargin=0.5cm}
\pdfcompresslevel=9

\begin{document}
	
	\textbf{Линейные ОДУ произвольного порядка с постоянными коэффициентами}\\

	Однородные линейные ОДУ имеют вид:
	\begin{equation}
		a_0y^{n}+a_1y^{n-1}+...+a_{n-1}y'+a_ny=0
	\end{equation}

	Метод решения: составить характеристическое уравнение:
	\begin{equation}
		a_0\lambda^n+a_1\lambda^{n-1}+...+a_{n-1}\lambda+a_n = 0
	\end{equation}

	Найти все его корни.\\

	Каждый корень характеристического уравнения добавляет в общее решение уравнения $(1)$ одно слагаемое.\\

	В случае простого вещественного корня:
	\begin{equation}
		C_ie^\lambda_i x
	\end{equation}

	В случае вещественного корня кратности $k$:
	\begin{equation}
		(C_{m+1}+C_{m+2}x+C_{m+3}x^2+...+C_{m+k}x^{k-1})e^{\lambda x}
	\end{equation}

	Для каждой пары комплексных простых корней $\lambda = \alpha\pm\beta i$:
	\begin{equation}
		C_{m+1}e^{\alpha x}\cos\beta x + C_{m+2}e^{\alpha x}\sin\beta x
	\end{equation}

	Для каждой пары комплексных корней кратности $k$:
	\begin{equation}
		P_{k-1}e^{\alpha x}\cos\beta x + Q_{k-1}e^{\alpha x}\sin\beta x		
	\end{equation}

	(Многочлены $P$ и $Q$ имеют вид многочлена в скобках $(4)$)



	
\end{document}
