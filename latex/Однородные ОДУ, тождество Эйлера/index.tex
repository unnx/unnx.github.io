\documentclass{article}
\usepackage[utf8]{inputenc}
\usepackage[russian]{babel}
\usepackage{cmap}
\usepackage{amsfonts,amsmath}
\usepackage{geometry}
\usepackage{fixint}
\usepackage{rumathgrk1}
\geometry{verbose,a4paper,tmargin=1cm,bmargin=1.5cm,lmargin=0.5cm,rmargin=0.5cm}
\pdfcompresslevel=9

\begin{document}
	
	\textbf{Однородные ОДУ, тождество Эйлера}\\

	Однородные ОДУ имеют вид:
	\begin{equation}
		y' = f(\frac{y}{x})
	\end{equation}

	Метод решения: замена $\frac{y}{x}=z$. $z=z(x)$ -- новая функция. Затем сведение к ОДУ с разделяющимися переменными:
	\begin{equation}
		\frac{y}{x}=z \Rightarrow y = zx \Rightarrow y' = z'x + z
	\end{equation}
	\begin{equation}
		z'x = f(z) - z
	\end{equation}

	Разделение переменных, решение и обратная подстановка. Интегральные кривые образуют множество подобных кривых, проходящих через точку $(0,0)$.\\

	Функция $f(x,y)$ называется однородной степени $m$, если для нее выполняется:
	\begin{equation}
		f(tx,ty) = t^mf(x,y)
	\end{equation}

	Здесь $m\in\mathbb{Z}$, либо $m=\frac{p}{2q+1}$, где $p, q \in\mathbb{Z}$. Однородные функции удовлетворяют тождеству Эйлера:
	\begin{equation}
		x\frac{\partial f}{\partial x} + y\frac{\partial f}{\partial y} = mf
	\end{equation}

	Для доказательства -- продифференцировать $(4)$ и положить $t=1$.\\


	Если $f(x,y)$ однородна при $t>0$, то она называется положительно однородной.

	Рассмотрим уравнение:
	\begin{equation}
		P(x,y)dx + Q(x,y)dy = 0
	\end{equation}

	Если $P$ и $Q$ -- однородные функции одной степени, то $(6)$ является однородным ОДУ.
\end{document}
