\documentclass{article}
\usepackage[utf8]{inputenc}
\usepackage[russian]{babel}
\usepackage{cmap}
\usepackage{amsfonts,amsmath}
\usepackage{geometry}
\usepackage{fixint}
\usepackage{rumathgrk1}
\geometry{verbose,a4paper,tmargin=1cm,bmargin=1.5cm,lmargin=0.5cm,rmargin=0.5cm}
\pdfcompresslevel=9

\begin{document}
	
	\textbf{Индуктивность}\\

	Если магнитный поток через замкнутый контур изменяется со временем, то в этом контуре индуцируется ток. Аналогично, любой контур с током создает магнитное поле, просто потому, что является проводником с током. Чтобы связать магнитный поток через контур с током в нем, вводится коэффициент пропорциональности, называемый индуктивностью контура $L$:
	\begin{equation}
		\Phi = LI
	\end{equation}

	Для всевозможных конструкций, представимых в виде множества контуров (напр. соленоидов, тороидов), вводится также величина полного магнитного потока, равная числу витков $N$, помноженному на поток через один виток $\Phi$:
	\begin{equation}
		\Psi = N\Phi
	\end{equation}

	Таким образом, индуктивность такой конструкции:
	\begin{equation}
		L = \frac{\Psi}{I}
	\end{equation}

	Индуктивность может зависеть от разных величин. Покажем это. Распишем ЭДС индукции:
	\begin{equation}
		\varepsilon_i=-\frac{d\Psi}{dt}=-\frac{d(LI)}{dt}=-(L\frac{\partial I}{\partial t}+I\frac{\partial L}{\partial t})
	\end{equation}

	Интуитивно ясно (доказывать не будем), что характеристики контура, в т.ч. индуктивность, практически не зависят от времени при ничтожно слабом намагничивании, т.е. в отсутствие ферромагнетиков. В этом случае:
	\begin{equation}
		\varepsilon_i=-L\frac{\partial I}{\partial t}
	\end{equation}

	Если ферромагнетики есть, можно, проведя некоторые преобразования, показать зависимость индуктивности от силы тока в контуре:
	\begin{equation}
		\varepsilon_i=-(L\frac{\partial I}{\partial t}+I\frac{\partial L}{\partial t})=-(L+I\frac{\partial L}{\partial I})\frac{\partial I}{\partial t}
	\end{equation}

	Выходит, что индуктивность действительно может зависеть от разных величин.

\end{document}
