\documentclass{article}
\usepackage[utf8]{inputenc}
\usepackage[russian]{babel}
\usepackage{cmap}
\usepackage{amsfonts,amsmath}
\usepackage{geometry}
\usepackage{fixint}
\usepackage{rumathgrk1}
\geometry{verbose,a4paper,tmargin=1cm,bmargin=1.5cm,lmargin=0.5cm,rmargin=0.5cm}
\pdfcompresslevel=9

\begin{document}
	
	\textbf{Мощность тока, закон Джоуля-Ленца}\\

	Электростатические силы при перемещении заряда по проводнику совершают работу:
	\begin{equation}
		A = Uq = UIt
	\end{equation}

	Разделив работу на время, получим мощность:
	\begin{equation}
		A = UI = (\varphi_1-\varphi_2)I + \mathcal{E}_{12}I
	\end{equation}

	Удельная мошность выражается через объем:
	\begin{equation}
		P_{\text{уд}} = \frac{\Delta P}{\Delta V}
	\end{equation}

	Получим выражение для мощности тока. Мощность механической силы равна скалярному произведению силы и скорости тела, на которое она действует. Сила $e(\vec E + \vec E*)$ развивает мощность:
	\begin{equation}
		P' = (e(\vec E + \vec E^*);\vec u)
	\end{equation}

	Мощность $\Delta P$, развиваемая в объеме $\Delta V$ получаем, умножив $P'$ на число зарядов в этом объеме $n\Delta V$:
	\begin{equation}
		\Delta P = P'n\Delta V = (e(\vec E + \vec E^*);\vec u) n\Delta V = (\vec j,\vec E + \vec E^*)\Delta V
	\end{equation}

	Отсюда:
	\begin{equation}
		P_{\text{уд}} = \vec j(\vec E + \vec E^*)
	\end{equation}

	Джоуль и Ленц установили выражение для количества теплоты, выделяющегося при прохождении тока $I$ через разность потенциалов $U$ за время $t$:
	\begin{equation}
		Q = UIt
	\end{equation}

	Отсюда:
	\begin{equation}
		Q = RI^2 t
	\end{equation}


\end{document}
