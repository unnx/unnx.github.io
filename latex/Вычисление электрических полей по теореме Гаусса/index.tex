\documentclass{article}
\usepackage[utf8]{inputenc}
\usepackage[russian]{babel}
\usepackage{cmap}
\usepackage{amsfonts,amsmath}
\usepackage{geometry}
\usepackage{fixint}
\usepackage{rumathgrk1}
\geometry{verbose,a4paper,tmargin=1cm,bmargin=1.5cm,lmargin=0.5cm,rmargin=0.5cm}
\pdfcompresslevel=9

\begin{document}
	
	\textbf{Вычисление электрических полей по теореме Гаусса}\\

	Теорема Гаусса для вектора $\vec E$ гласит: поток вектора $\vec E$ через замкнутую поверхность равен заряду, заключенному в объеме, ограниченном этой поверхностью, деленному на $\varepsilon_0$:
	\begin{equation}
		\oint{\vec E d\vec S} = \frac{q}{\varepsilon_0}
	\end{equation}

	Если часть поверхности, поток через которую не равен нулю, является эквипотенциальной, модуль $\vec E$ можно вынести из-под знака интеграла. Воспользуемся этим для вычисления полей разных заряженных фигур.\\

	Для бесконечной плоскости с поверхностной плотностью $\sigma$ эквипотенциальными поверхностями являются плоскости, параллельные данной. В качестве замкнутой поверхности удобно взять прямой цилиндр, образующие которого перпендикулярны заряженной плоскости. Тогда:
	\begin{equation}
		E \oint{d \vec S} = \frac{q}{\varepsilon_0}
	\end{equation}
	\begin{equation}
		E = \frac{q}{2S\varepsilon_0}
	\end{equation}
	\begin{equation}
		E = \frac{\sigma}{2\varepsilon_0}
	\end{equation}

	Легко видеть, что в случае двух бесконечных разноименно заряженных плоскостей напряженность поля между ними равна $\frac{\sigma}{\varepsilon_0}$, а снаружи поле отсутствует.\\

	Для заряженной сферы эквипотенциальными поверхностями являются сферы. Если радиус выбранной поверхности меньше радиуса заряженной сферы, то такая поверхность не охватывает зарядов, поэтому $E=0$. Допустим, ее радиус $r$ больше радиуса заряженной сферы $R$:
	\begin{equation}
		E4\pi r^2 = \frac{q}{\varepsilon_0}
	\end{equation}
	\begin{equation}
		E = \frac{1}{4\pi\varepsilon_0}\frac{q}{r^2}
	\end{equation}

	Таким образом, внутри сферы поле отсутствует, а снаружи соответствует полю точечного заряда, сконцентрированного в центре сферы.\\

	Случай шара c объемной плотностью заряда $\rho$ аналогичен предыдущему с той разницей, что сфера радиуса $r<R$ тоже охватывает заряды:
	\begin{equation}
		E4\pi r^2 = \frac{\rho \frac{4}{3}\pi r^3}{\varepsilon_0}
	\end{equation}
	\begin{equation}
		E = \frac{\rho r}{3\varepsilon_0}, \;\; r\leq R
	\end{equation}

	$\rho$ можно заменить на $\frac{3q}{4\pi R^3}$:
	\begin{equation}
		E = \frac{1}{4\pi\varepsilon_0}\frac{q}{R^3}r, \;\; r\leq R
	\end{equation}	

	Снаружи шар создает поле, совпадающее с полем сферы того же заряда и радиуса:
	\begin{equation}
		E = \frac{1}{4\pi\varepsilon_0}\frac{q}{r^2}, \;\; r>R
	\end{equation}


\end{document}
