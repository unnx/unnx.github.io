\documentclass{article}
\usepackage[utf8]{inputenc}
\usepackage[russian]{babel}
\usepackage{cmap}
\usepackage{amsfonts,amsmath}
\usepackage{geometry}
\usepackage{fixint}
\usepackage{rumathgrk1}
\geometry{verbose,a4paper,tmargin=1cm,bmargin=1.5cm,lmargin=0.5cm,rmargin=0.5cm}
\pdfcompresslevel=9

\begin{document}

	\textbf{Фотоны, тормозное излучение, фотоэффект, эффект Комптона}\\

	Тормозное излучение возникает при торможении быстрых электронов. Его спектр имеет коротковолновую границу, что еще раз подтверждает квантовую природу излучения.\\

	Мощность тормозного излучения $P$ пропорциональна квадрату заряда электрона и квадрату его ускорения:
	\begin{equation}
		P\sim e^2 a^2
	\end{equation}

	Соответственно, если ускорение электрона остается постояным в течение всего времени торможения $\tau$, то мощность излучения также будет постоянной и за время торможения электрон излучит энергию:
	\begin{equation}
		E = P\tau \sim e^2a^2\tau = e^2 v_0^2 / \tau
	\end{equation}

	$v_0$ -- начальная скорость электрона\\

	Если электрон разгоняется напряжением $U$, то ввиду квантования энергии излучения, справедливо неравенство:
	\begin{equation}
		\hbar\omega \leq eU
	\end{equation}

	Отсюда $\omega_{max} = \frac{eU}{\hbar}$, и:
	\begin{equation}
		\lambda_{min} = \frac{2\pi c}{\omega_{max}} = \frac{(2\pi\hbar c/e)}{U}
	\end{equation}

	При облучении различных материалов появляются свободные электроны. Это явление называется фотоэффектом. Его обычно наблюдают в специальной вакуумной трубке с двумя электродами, между которыми поддерживается напряжение $U$. При достижении им значения $U_\text{з}$, ток между электродами прекращается. Такое напряжение называется запирающим. Энергия, которую оно отбирает у электронов равна максимальной кинетической энергии, которую они приобретают за счет фотоэффекта:
	\begin{equation}
		\frac{1}{2}mv_{max}^2 = eU_\text{з} 
	\end{equation}

	Таким образом, освободившийся под действием излучения электрон, вообще говоря, может иметь ненулевую скорость. Связь между энергией свободного электрона и частотой излучения видна в формуле Эйнштейна для фотоэффекта:
	\begin{equation}
		\hbar\omega = \frac{1}{2}mv_{max}^2 + A
	\end{equation}

	Где $A$ -- т.н. работа выхода, т.е. энергия, которая тратится на то, чтобы электрон покинул тело.\\

	Таким образом, фотоэффект возможен при условии:
	\begin{equation}
		\omega \geq \omega_0 = A/\hbar
	\end{equation}

	Частота $\omega_0$ (или соответствующая ей длина волны $\lambda_0=\frac{2\pi\hbar c}{A}$) называется красной границей фотоэффекта.\\

	Током насыщения $I_\text{н}$ называется ток, образованный всеми электронами, вылетевшими под действием фотоэффекта в единицу времени. Само собой, он пропорционален световому потоку:
	\begin{equation}
		I_\text{н}\sim\Phi
	\end{equation}

	Однако существует т.н. многофотонный фотоэффект, заключающийся в том, что выбиваемый электрон получает энергию не от одного, а от многих фотонов. Формула Эйнштейна преобразуется:
	\begin{equation}
		N\hbar\omega = \frac{1}{2}mv_{max}^2 + A
	\end{equation}

	Ток насыщения в этом случае пропорционален световому потоку в соответствующей степени:
	\begin{equation}
		I_\text{н} = \Phi^N
	\end{equation}

	Из текста выше очевидно, что фотон обладает энергией. Он является квантом света, передающимся электромагнитной волной, а у волны также есть и импульс. Следовательно, фотон должен обладать импульсом.\\

	Рассмотрим две системы отсчета $K$ и $K'$, движущиеся друг относительно друга со скоростью $\vec v_0$. Оси $x$ и $x'$ направлены вдоль $\vec v_0$. Пусть в направлении этих осей летит фотон. Его энергия в этих системах равна $\hbar\omega$ и $\hbar\omega'$ соответственно. Эти частоты связаны соотношением:
	\begin{equation}
		\omega' = \omega\frac{1-\frac{v_0}{c}}{\sqrt{1-\frac{v_0^2}{c^2}}}
	\end{equation}

	Поэтому:
	\begin{equation}
		E'=E\frac{1-\frac{v_0}{c}}{\sqrt{1-\frac{v_0^2}{c^2}}}
	\end{equation}

	При переходе от $E$ к $E'$ энергия преобразуется по формуле:
	\begin{equation}
		E' = \frac{E-v_0 p_x}{\sqrt{1-\frac{v_0^2}{c^2}}}
	\end{equation}

	Таким образом:
	\begin{equation}
		E(1-\frac{v_0}{c}) = E - v_0 p
	\end{equation}

	Отсюда:
	\begin{equation}
		p = \frac{E}{c} = \frac{\hbar\omega}{c}
	\end{equation}

	Можно записать импульс фотона в векторном виде:
	\begin{equation}
		\vec p = \hbar\vec k
	\end{equation}

	Корпускулярные свойства фотонов проявляются в т.н. эффекте Комптона.\\

	Артур Комптон обнаружил, что при рассеянии рентгеновских лучей на веществе помимо излучения первоначальной длины волны присутствуют лучи большей длины волны, причем разность между этими длинами зависит только от угла между исходным и рассеянным лучами.\\

	Пусть на первоначально покоящийся свободный электрон падает фотон с энергией $\hbar\omega$ и импульсом $\hbar\vec k$. Энергия электрона до столкновения равна $mc^2$, импульс равен нулю. После столкновения электрон будет обладать импульсом $\vec p$ и энергией, равной $c\sqrt{p^2 + m^2c^2}$. Энергия и импульс фотона также изменятся и станут равными $\hbar\omega'$ и $\hbar\vec k'$. Из законов сохранения энергии следует:
	\begin{equation}
		\hbar\omega + mc^2 = \hbar\omega' + c\sqrt{p^2+m^2c^2}
	\end{equation}
	\begin{equation}
		\hbar\vec k = \vec p + \hbar\vec k'
	\end{equation}

	Разделим первое равенство на $c$:
	\begin{equation}
		\sqrt{p^2+m^2c^2} = \hbar(k-k') + mc
	\end{equation}

	Возведем в квадрат:
	\begin{equation}
		p^2 = \hbar^2(k^2+k'^2-2kk')+2\hbar mc(k-k')
	\end{equation}

	Получим $p^2$ из $(18)$:
	\begin{equation}
		p^2 = \hbar^2(\vec k - \vec k')^2 = \hbar^2(k^2+k'^2-2kk'\cos\vartheta)
	\end{equation}

	$\vartheta$ -- угол между первоначальным и рассеянным пучком.\\

	Совместив два выражения, получим:
	\begin{equation}
		mc(k-k') = \hbar k k'(1-\cos\vartheta)
	\end{equation}

	Умножим на $\frac{2\pi}{mckk'}$:
	\begin{equation}
		\frac{2\pi}{k'}-\frac{2\pi}{k}=\frac{2\pi\hbar}{mc}(1-\cos\vartheta)
	\end{equation}

	Получаем:
	\begin{equation}
		\Delta\lambda = \lambda' - \lambda = \frac{2\pi\hbar}{mc}(1-\cos\vartheta)
	\end{equation}

	Множитель перед скобками называется комптоновской длиной волны и обозначается $\lambda_C$
	
\end{document}
