\documentclass{article}
\usepackage[utf8]{inputenc}
\usepackage[russian]{babel}
\usepackage{cmap}
\usepackage{amsfonts,amsmath}
\usepackage{geometry}
\usepackage{fixint}
\usepackage{rumathgrk1}
\geometry{verbose,a4paper,tmargin=1cm,bmargin=1.5cm,lmargin=0.5cm,rmargin=0.5cm}
\pdfcompresslevel=9

\begin{document}
	
	\textbf{Стоячие упругие волны}\\

	Стоячие волны -- это волны, в которых не осуществляется перенос энергии. Они образуются при наложении нескольких волн друг на друга.\\

	Представим две упругие одномерные волны:
	\begin{equation}
		\xi_1 = a\cos(\omega t - kx + \varphi_1)
	\end{equation}
	\begin{equation}
		\xi_2 = a\cos(\omega t - kx + \varphi_2)
	\end{equation}

	При сложении получается:
	\begin{equation}
		\xi = 2a\cos(kx+\frac{\varphi_2-\varphi_1}{2})\cos(\omega t + \frac{\varphi_1+\varphi_2}{2})
	\end{equation}

	При соответствующем выборе начал отсчета времени и координаты избавимся от смещения фаз:
	\begin{equation}
		\xi = 2a\cos(kx)\cos(\omega t)
	\end{equation}

	Это уравнение стоячей волны. Из него следует, что в стоячей волне амплитуда зависит от координаты:
	\begin{equation}
		\text{амплитуда} = |2a\cos(kx)|=|2a\cos(2\pi\frac{x}{\lambda})|
	\end{equation}

	Пучностями называются участки волны с максимальной амплитудой. Чтобы найти их координаты в нашем случае, требуется найти максимум косинуса (далее $n\in\mathbb{Z}$):
	\begin{equation}
		2\pi\frac{x}{\lambda}=\pi n
	\end{equation}

	Отсюда:
	\begin{equation}
		x_\text{пучн.} = n\frac{\lambda}{2}
	\end{equation}

	Узлами называются участки волны с нулевой амплитудой. В стоячей волне в узлах колебания отсутствуют. Найдем условия равенства нулю амплитуды:
	\begin{equation}
		2\pi\frac{x}{\lambda}=\frac{\pi}{2}+\pi n=(n+\frac{1}{2})\pi
	\end{equation}

	Отсюда:
	\begin{equation}
		x_\text{узл.} = (n+1)\frac{\lambda}{2}
	\end{equation}

	Легко видеть, что:\\

	1) Расстояние между двумя ближайшими узлами или пучностями равно $\frac{\lambda}{2}$.\\

	2) Расстояние между ближайшими пучностью и узлом равно $\frac{\lambda}{4}$\\

	Рассмотрим колебания струны, закрепленной в концах. Из закрепленности концов следует, что в них должны наблюдаться узлы. Следовательно, в этой струне будут возбуждаться колебания, половина длины волны которых укладывается на длине струны целое число раз:
	\begin{equation}
		l = n\frac{\lambda}{2}
	\end{equation}

	Отсюда (далее $n\in\mathbb{N}$):
	\begin{equation}
		\lambda_n = \frac{2l}{n}
	\end{equation}

	Легко найти частоту:
	\begin{equation}
		\nu_n = \frac{v}{2l}n
	\end{equation}

	Мы нашли длину волны и частоту $n$-ной гармоники струны: т.е. колебания, при котором половина длины получаемой волны укладывается между ближайшими узлами $n\in\mathbb{N}$ раз.


\end{document}
