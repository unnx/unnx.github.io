\documentclass{article}
\usepackage[utf8]{inputenc}
\usepackage[russian]{babel}
\usepackage{cmap}
\usepackage{amsfonts,amsmath}
\usepackage{geometry}
\usepackage{fixint}
\usepackage{rumathgrk1}
\geometry{verbose,a4paper,tmargin=1cm,bmargin=1.5cm,lmargin=0.5cm,rmargin=0.5cm}
\pdfcompresslevel=9

\begin{document}
	
	\textbf{Интенсивность звука и громкость}\\

	Интенсивностью звука называется среднее по времени значение модуля вектора Умова:
	\begin{equation}
		I = <j> = \frac{1}{2}\rho a^2\omega^2v
	\end{equation}

	Человеческое ухо воспринимает звуки разной частоты с разной силой, даже если поток энергии для них одинаков. Порогом слышимости называется кривая на множестве $I(\nu)$ (оно называется звуковым спектром). Если звук представляется точкой, лежащей ниже этой границы, человек его не услышит. Минимум этой кривой обозначают константой $I_0=10^{-12}\;\text{Вт}/\text{м}$ -- это наименьшее значение среднего потока энергии звука, который может услышать человек (оно соответствует частоте около $2000\;\text{Гц}$). Аналогичная кривая ограничивает звуковой спектр сверху: она называется порогом болевого ощущения. Звук, отображаемый на спектре точками выше этого порога, не вызывает в ухе человека ничего, кроме боли.\\

	Неудобно пользоваться шкалой интенсивности звука: ухо человека ощущает линейное увеличение интенсивности, в то время как реальное увеличение происходит в геометрической прогрессии. Человеку больше подходит искусственно созданная шкала громкости. \\

	Уровень громкости в децибеллах (дБ) определяется как десятичный логарифм отношения интенсивности исследуемого звука к минимуму порога слышимости, помноженный на $10$:
	\begin{equation}
		L = 10\lg\frac{I}{I_0}
	\end{equation}

	Отношение любых двух интенсивностей может быть выражено в децибеллах:
	\begin{equation}
		L = 10\lg\frac{I_1}{I_2}
	\end{equation}
\end{document}
