\documentclass{article}
\usepackage[utf8]{inputenc}
\usepackage[russian]{babel}
\usepackage{cmap}
\usepackage{amsfonts,amsmath}
\usepackage{geometry}
\usepackage{fixint}
\usepackage{rumathgrk1}
\geometry{verbose,a4paper,tmargin=1cm,bmargin=1.5cm,lmargin=0.5cm,rmargin=0.5cm}
\pdfcompresslevel=9

\begin{document}

	\textbf{Квантование момента импульса, спин}\\

	Решение уравнения на собственные значения и собственные функции оператора момента импульса дает следующий набор возможных дискретных значений для модуля момента импульса:
	\begin{equation}
		M = \hbar\sqrt{l(l+1)}\;\;\;(l=0,1,2,...)
	\end{equation}

	Рассмотрим проекцию момента на выделенную ось $M_z$. Уравнение на собственные значения и собственные функции оператора имеет вид:
	\begin{equation}
		\hat M_z\psi = M_z\psi
	\end{equation}

	В полярных координатах $(r,\vartheta,\varphi)$ оператор проекции момента импульса на полярную ось $z$ имеет вид:
	\begin{equation}
		\hat M_z = -i\hbar\frac{\partial}{\partial\varphi}
	\end{equation}

	Уравнение Шредингера:
	\begin{equation}
		-i\hbar\frac{\partial\psi}{\partial\varphi} = M_z\psi
	\end{equation}

	После подстановки $\psi = e^{\alpha\varphi}$:
	\begin{equation}
		-i\hbar\alpha = M_z
	\end{equation}

	Отсюда:
	\begin{equation}
		\psi = Ce^{(i\frac{M_z}{\hbar})\varphi}
	\end{equation}

	Условие однозначности выражается в том, что $\psi(\varphi+2\pi)=\psi(\varphi)$. Поэтому $\hat M_z$ обладает дискретным спектром:
	\begin{equation}
		M_z = m\hbar\;\;\;(m=0,\pm 1,\pm 2,...)
	\end{equation}

	$m$ называется магнитным квантовым числом.\\

	Суммарный модуль момента импульса системы из двух частиц, как и любой другой модуль момента импульса, определяется выражением:
	\begin{equation}
		M = \hbar\sqrt{L(L+1)},\;\;\;L=l_1+l_2,\;l_1+l_2-1,\;...,|l_1-l_2|
	\end{equation}

	Проекция результирующего момента на некоторую ось определяется выражением:
	\begin{equation}
		M_z = m_L\hbar\;\;\;(m_L=0,\pm 1,\pm 2,...,\pm L)
	\end{equation}

	Электрон имеет собственные механический и магнитный моменты $M_s$ и $\mu_s$ соответственно. Между ними есть связь:
	\begin{equation}
		\frac{\mu_s}{M_s} = -\frac{e}{m_e c}
	\end{equation}

	Модуль собственного момента импульса электрона определяется через спиновое число, равное для электрона $1/2$:
	\begin{equation}
		M_s = \hbar\sqrt{s(s+1)} = \hbar\sqrt{\frac{1}{2}\frac{3}{2}}=\frac{1}{2}\hbar\sqrt{3}
	\end{equation}

	Проекция спина на заданное направление может принимать значения:
	\begin{equation}
		M_{sz}=m_s\hbar\;\;\;(m_s=\pm s = \pm 1/2)
	\end{equation}

	Воспользуемся магнитомеханическим соотношением выше, чтобы найти $\mu_s$:
	\begin{equation}
		\mu_s = -\frac{e}{m_e c}M_s = -\frac{e\hbar}{m_e c}\sqrt{s(s+1)} = -2\mu_\text{Б}\sqrt{s(s+1)}=-\mu_\text{Б}\sqrt{3}
	\end{equation}

	Величина $\mu_\text{Б}=-\frac{e\hbar}{2m_ec}$ называется магнетоном Бора.
	
\end{document}
