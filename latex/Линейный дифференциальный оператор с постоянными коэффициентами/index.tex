\documentclass{article}
\usepackage[utf8]{inputenc}
\usepackage[russian]{babel}
\usepackage{cmap}
\usepackage{amsfonts,amsmath}
\usepackage{geometry}
\usepackage{fixint}
\usepackage{rumathgrk1}
\geometry{verbose,a4paper,tmargin=1cm,bmargin=1.5cm,lmargin=0.5cm,rmargin=0.5cm}
\pdfcompresslevel=9

\begin{document}
	
	\textbf{Линейный дифференциальный оператор с постоянными коэффициентами}\\

	Имеет вид:
	\begin{equation}
		l(D) = \sum_{k=0}^n a_{n-k}D^k,
	\end{equation}

	Где $D$ -- оператор дифференцирования $\frac{d}{dx}$, а $a_{n-k}..a_0$ -- постоянные коэффициенты, вещественные или комплексные.\\

	Линейный дифференциальный оператор $D$ имеет два свойства:\\

	1) Умножение на действительную постоянную $c\in\mathbb{R}$:
	\begin{equation}
		cl(D)[y] = l(D)[cy]
	\end{equation}
	\begin{equation}
		cl(D)[y] = \sum_{k=0}^n ca_{n-k}D^k[y] = \sum_{k=0}^n a_{n-k}D^k[cy] 
	\end{equation}

	2) Сложение двух операторов:
	\begin{equation}
		l(D)[y_1+y_2] = l(D)[y_1] + l(D)[y_2]
	\end{equation}
	\begin{equation}
		l(D)[y_1+y_2] = \sum_{k=0}^n ca_{n-k}D^k[y_1 + y_2] = \sum_{k=0}^n ca_{n-k}D^k[y_1] + \sum_{k=0}^n ca_{n-k}D^k[y_2]
	\end{equation}

\end{document}
