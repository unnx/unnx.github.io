\documentclass{article}
\usepackage[utf8]{inputenc}
\usepackage[russian]{babel}
\usepackage{cmap}
\usepackage{amsfonts,amsmath}
\usepackage{geometry}
\usepackage{fixint}
\usepackage{rumathgrk1}
\geometry{verbose,a4paper,tmargin=1cm,bmargin=1.5cm,lmargin=0.5cm,rmargin=0.5cm}
\pdfcompresslevel=9

\begin{document}
	
	\textbf{Опыт Милликена по определению заряда электрона}\\

	В 1909 г. Милликен довольно точно определил заряд электрона.\\

	Между двумя горизонтальными пластинами конденсатора он разбрызгивал наэлектризованные капельки масла. Меняя знак и величину напряжения на пластинах, можно было добиться неподвижности капель. Это равновесие наступает при условии:
	\begin{equation}
		P' = e'E
	\end{equation}

	Здесь $P'$ -- результирующая силы тяжести и архимедовой силы:
	\begin{equation}
		P' = \frac{4}{3}\pi r^3 (\rho - \rho_0)g
	\end{equation}

	$\rho$ -- плотность капельки, $\rho'$ -- плотность воздуха.\\

	Зная радиус капли, можно найти $e'$. Его можно найти через скорость равномерного падения капли $v_0$. Условие равномерного падения:
	\begin{equation}
		P' = 6\pi\eta r v_0
	\end{equation}

	(Движение капли наблюдалось с помощью микроскопа)\\

	Установить равновесие очень трудно, поэтому включалось дополнительное поле, под действием которого капелька начинала двигаться вверх с постоянной скоростью $v_E$. Условие равномерности движения:
	\begin{equation}
		P' + 6\pi\eta r v_E = e'E
	\end{equation}

	Раскрыв $P'$ и $r$, Милликен получил окончательное выражение:
	\begin{equation}
		e' = 9\pi\sqrt{\frac{2\eta^3 v_0}{(\rho-\rho_0)g}}\frac{v_0+v_E}{E}
	\end{equation}

	В эту формулу внесена поправка, учитывающая, что размеры капель сравнимы с длиной свободного пробега молекул воздуха.\\

	В ходе опыта была также доказана дискретность заряда.


\end{document}
