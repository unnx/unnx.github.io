\documentclass{article}
\usepackage[utf8]{inputenc}
\usepackage[russian]{babel}
\usepackage{cmap}
\usepackage{amsfonts,amsmath}
\usepackage{geometry}
\usepackage{fixint}
\usepackage{rumathgrk1}
\geometry{verbose,a4paper,tmargin=1cm,bmargin=1.5cm,lmargin=0.5cm,rmargin=0.5cm}
\pdfcompresslevel=9

\begin{document}
	
	\textbf{Ток смещения}\\

	Ротор $\vec H$ равен плотности макроскопических токов:
	\begin{equation}
		[\vec\nabla;\vec H] = \vec j
	\end{equation}

	Вектор $\vec j$ связан с плотностью сторонних зарядов уравнением непрерывности:
	\begin{equation}
		(\vec\nabla;\vec j) = -\frac{\partial\rho}{\partial t}
	\end{equation}

	Представим заряжающийся конденсатор. Рассмотрим одну из его пластин: по проводнику, соединенному с ней течет ток $I$, который прекращается после зарядки конденсатора. Выберем поверхность $S_1$, пересекающую провод, и другую поверхность $S_2$ -- между обкладками конденсатора. По теореме Стокса:
	\begin{equation}
		\oint_{l_1} \vec H d\vec l = \int_{S_1} \vec j d\vec S = I 
	\end{equation}
	\begin{equation}
		\oint_{l_2} \vec H d\vec l = \int_{S_2} \vec j d\vec S = 0 
	\end{equation}

	Поверхность $S_2$ не охватывает токов, поэтому циркуляция $\vec H$ по контуру, ограничивающему эту поверхность, равна нулю. Этого не может быть, т.к. напряженность поля между обкладками точно не равна нулю. \\

	Следовательно, в случае изменяющегося во времени магнитного поля, теорема о циркуляции $H$ не работает. \\

	Несоответствие также выражается в том, что если взять дивергенцию от левой и правой части уравнения $[\vec\nabla;\vec H] = \vec j$, слева всегда будет $0$, а справа -- необязательно. Это противоречит уравнению непрерывности.\\

	Максвелл ввел в правую часть уравнения $[\vec\nabla;\vec H] = \vec j$ дополнительное слагаемое $\vec j_\text{смещ}$:
	\begin{equation}
		[\vec\nabla;\vec H] = \vec j + \vec j_\text{смещ} = \vec j_\text{полн}
	\end{equation}

	Это слагаемое называется током смещения и имеет размерность плотности тока. Если $(\vec\nabla;\vec j_\text{смещ}) = -(\vec\nabla;\vec j)$, то мы действительно получим тождество, взяв дивергенцию от обеих частей уравнения $(5)$:
	\begin{equation}
		0 = (\vec\nabla;[\vec\nabla;\vec H]) = -(\vec\nabla;\vec j) + (\vec\nabla;\vec j) = 0
	\end{equation}

	Из уравнения непрерывности:
	\begin{equation}
		(\vec\nabla;\vec j) = \frac{\partial\rho}{\partial t}
	\end{equation}

	По теореме о потоке вектора $\vec D$:
	\begin{equation}
		(\vec\nabla;\vec D) = \rho
	\end{equation}
	\begin{equation}
		\frac{\partial}{\partial t}(\vec\nabla;\vec D) = \frac{\partial\rho}{\partial t}
	\end{equation}
	\begin{equation}
		\frac{\partial\rho}{\partial t} = (\vec\nabla;\frac{\partial\vec D}{\partial t})
	\end{equation}
	\begin{equation}
		(\vec\nabla;j_\text{смещ}) = (\vec\nabla;\frac{\partial\vec D}{\partial t})
	\end{equation}

	Окончательное выражение для $\vec j_\text{смещ}$:
	\begin{equation}
		\vec j_\text{смещ} = \frac{\partial \vec D}{\partial t}
	\end{equation}

\end{document}
