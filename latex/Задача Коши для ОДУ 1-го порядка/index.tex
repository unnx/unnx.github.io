\documentclass{article}
\usepackage[utf8]{inputenc}
\usepackage[russian]{babel}
\usepackage{cmap}
\usepackage{amsfonts,amsmath}
\usepackage{geometry}
\usepackage{fixint}
\usepackage{rumathgrk1}
\geometry{verbose,a4paper,tmargin=1cm,bmargin=1.5cm,lmargin=0.5cm,rmargin=0.5cm}
\pdfcompresslevel=9

\begin{document}
	
	\textbf{Задача Коши для ОДУ 1-го порядка}\\

	ОДУ 1-го порядка имеет вид:
	\begin{equation}
		y' = f(x,y)
	\end{equation}

	Здесь $f$ -- заданная функция:
	\begin{equation}
		f\in C(D),\;D\in \mathbb{R}^2
	\end{equation}

	Пусть дано начальное условие:
	\begin{equation}
		y(x_0) = y_0,\;(x_0,y_0)\in D
	\end{equation}

	Задача Коши состоит в нахождении частного решения ОДУ $(1)$, которое удовлетворяло бы начальному условию, т.е. найти $y=y(x)$, такую, что:
	\begin{equation}
		y' = f(x,y)
	\end{equation}
	\begin{equation}
		y(x_0) = y_0
	\end{equation}

	Если $y=y(x)$ -- решение $(1)$, то график этого решения называется интегральной кривой.\\

	Геометрическая интерпретация задачи Коши: найти интегральную кривую, проходящую через заданную точку.\\

	Теорема о существовании и единственности решения задачи Коши (ТСЕ):\\

	Пусть функция $f(x,y)$ и $\frac{\partial f}{\partial y}$ непрерывна в области $D \in \mathbb{R}^2$. Тогда $\forall (x_0,y_0)\in D$:\\

	1) $\exists$ окрестность $(x_0-\delta,x_0+\delta)$, в которой имеется решение $y=y(x)$ задачи Коши.\\

	2) Если $y=y_1(x)$ и $y=y_2(x)$ -- два решения задачи Коши, то $y_1(x)=y_2(x)$ в некоторой окрестности точки $x_0$.
\end{document}
