\documentclass{article}
\usepackage[utf8]{inputenc}
\usepackage[russian]{babel}
\usepackage{cmap}
\usepackage{amsfonts,amsmath}
\usepackage{geometry}
\usepackage{fixint}
\usepackage{rumathgrk1}
\geometry{verbose,a4paper,tmargin=1cm,bmargin=1.5cm,lmargin=0.5cm,rmargin=0.5cm}
\pdfcompresslevel=9

\begin{document}
	
	\textbf{ОДУ с разделяющимися переменными и сводящиеся к ним}\\

	ОДУ с разделяющимися переменными имеют вид:
	\begin{equation}
		y' = f(x)g(y),\;f(x)\in C(I),\;g(x)\in C(J)
	\end{equation}

	Метод решения -- разделение переменных и интегрирование:
	\begin{equation}
		\frac{dy}{dx} = f(x)g(y)
	\end{equation}
	\begin{equation}
		\frac{dy}{g(y)} = f(x)dx
	\end{equation}
	\begin{equation}
		\int_{y_0}^y \frac{dy}{g(y)} = \int_{x_0}^x f(x)dx + C
	\end{equation}

	Здесь $x_0\in I,\;y_0\in J$ -- фиксированные.\\

	Конечное уравнение:
	\begin{equation}
		G(y) = F(x) + C
	\end{equation}

	Формула $(5)$ может не содержать всех решений ОДУ $(1)$: если уравнение $g(y)=0$ имеет корень $y=y^*$, то $y=y^*$ -- решение ОДУ $(1)$.\\

	Задача Коши состоит в отыскании решения ОДУ $(1)$, удовлетворяющего условиям:
	\begin{equation}
		y' = f(x)g(y)
	\end{equation}
	\begin{equation}
		y(x_0) = y_0
	\end{equation}

	Если $g(y_0)\neq 0$, то задача Коши имеет единственное решение:
	\begin{equation}
		\int_{y_0}^y \frac{dy}{g(y)} = \int_{x_0}^x f(x)dx
	\end{equation}

	Если $g(y_0)=0$, то $y = y_0$ -- решение задачи Коши, но у этой задачи Коши могут быть и другие решения, если несобственный интеграл $\int_{y_0}^y$ сходится.\\

	К ОДУ с разделяющимися переменными, например, сводятся:\\

	Однородные ОДУ:
	\begin{equation}
		y' = f(\frac{y}{x})
	\end{equation}

	Линейные ОДУ первого порядка:
	\begin{equation}
		y' + a(x)y = b(x)
	\end{equation}
\end{document}
