\documentclass{article}
\usepackage[utf8]{inputenc}
\usepackage[russian]{babel}
\usepackage{cmap}
\usepackage{amsfonts,amsmath}
\usepackage{geometry}
\usepackage{fixint}
\usepackage{rumathgrk1}
\geometry{verbose,a4paper,tmargin=1cm,bmargin=1.5cm,lmargin=0.5cm,rmargin=0.5cm}
\pdfcompresslevel=9

\begin{document}
	
	\textbf{Стоячие волны в прямоугольном резонаторе с идеально проводящими стенками}\\

	Внутри такого резонатора бегут волны со следующими проекциями волновых векторов:

	\begin{equation}
		\{\pm k_x, \pm k_y, \pm k_z\}
	\end{equation}

	(Легко видеть, что их 8 штук)\\

	Если при отражении от стенок фаза волн не меняется, уравнение стоячей волны имеет вид:
	\begin{equation}
		\xi = \sum_{i=1}^8 \xi_i = 8A\cos(k_x x)\cos(k_y y)\cos(k_z z)\cos(\omega t)
	\end{equation}

	В случае изменения фазы на $\pi$ при отражении:
	\begin{equation}
		\xi = \sum_{i=1}^8 \xi_i = 8A\cos(k_x x + \frac{\pi}{2})\cos(k_y y - \frac{\pi}{2})\cos(k_z z + \frac{\pi}{2})\cos(\omega t + \frac{\pi}{2})
	\end{equation}

	Чтобы во всех вершинах резонатора амплитуда имела одинаковое значение, необходимо выполнение условий:
	\begin{equation}
		k_x = \frac{\pi}{a}n_1;\;\;k_y = \frac{\pi}{b}n_2;\;\;k_z = \frac{\pi}{c}n_3
	\end{equation}

	$a, b, c$ -- размеры резонатора, $n_1, n_2, n_3$ -- натуральные числа.\\

	В $k$-пространстве каждой стоячей волне соответствует точка с такими координатами. На долю каждой точки приходится объем $\frac{\pi^3}{abc}$, поэтому плотность точек равна $V/\pi^3$. Число точек, у которых проекции волновых векторов заключены в пределах $(k_x,k_x+dx)$, $(k_y,k_y+dy)$, $(k_z,k_z+dz)$ равно:

	\begin{equation}
		dN_{k_x,k_y,k_z} = \frac{V}{\pi^3}dk_x dk_y dk_z
	\end{equation}

	Но точки, соответствующие стоячим волнам, лежат только в первом октанте. Поэтому число стоячих волн в шаровом слое радиуса $k$ и ширины $dk$ равно:
	\begin{equation}
		dN_k = \frac{V}{\pi^3}\frac{1}{8}4\pi k^2 dk = V\frac{k^2 dk}{2\pi^2}
	\end{equation}

	Поскольку $k=\omega/v$ и $dk = d\omega/v$, можно найти число стоячих волн с частотами от $\omega$ до $\omega+d\omega$ в резонаторе объема $V$:
	\begin{equation}
		dN_\omega = V\frac{\omega^2 d\omega}{2\pi^2 v^3}
	\end{equation}

	Или число таких волн в единице объема:
	\begin{equation}
		dn_\omega = \frac{\omega^2 d\omega}{2\pi^2 v^3}
	\end{equation}

	

\end{document}
