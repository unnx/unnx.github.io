\documentclass{article}
\usepackage[utf8]{inputenc}
\usepackage[russian]{babel}
\usepackage{cmap}
\usepackage{amsfonts,amsmath}
\usepackage{geometry}
\usepackage{fixint}
\usepackage{listings}
\usepackage{rumathgrk1}
\geometry{verbose,a4paper,tmargin=1cm,bmargin=1.5cm,lmargin=0.5cm,rmargin=0.5cm}
\usepackage{color}
\lstset{ %
  backgroundcolor=\color{white},
  basicstyle=\footnotesize,
  breakatwhitespace=false,
  breaklines=true,
  deletekeywords={...},
  escapeinside={\%*}{*)},
  extendedchars=true,
  captionpos=none,
  keepspaces=false,
  morekeywords={*,...},
  showspaces=false,
  showstringspaces=false, 
  tabsize=2,
  title=\lstname
}
\pdfcompresslevel=9

\begin{document}
	
	\textbf{Haskell: простейшие операции со списками}\\

	\textbf{(:)} -- конструктор списка; равен списку, составленному из некоторого значения и другого списка со значениями того же типа (либо пустого), посредством вставки данного значения в начало существующего списка\\

	\textit{Тип}:
	\begin{lstlisting}
	(:) :: a -> [a] -> [a]
	\end{lstlisting}

	\null\textit{Пример}. Следующие выражения эквивалентны $True$:
	\begin{lstlisting}
	1:[2, 3, 4, 5] == [1, 2, 3, 4, 5]
	1:2:[3, 4, 5] == [1, 2, 3, 4, 5]
	1:[] == [1]
	's':[] == "s"
	[]:[] == [[]]
	[True]:[[False],[False]] == [[True],[False],[False]]
	[1, 2, 3] : [[4, 5, 6]] == [[1, 2, 3],[4, 5, 6]]
	[1, 2, 3] : [[4]] == [[1, 2, 3],[4]]
	\end{lstlisting}

	\null\textbf{(!!)} -- равна $n$-ному значению списка (нумерация с нуля)\\

	\textit{Тип}:
	\begin{lstlisting}
	(!!) :: [a] -> Int -> a
	\end{lstlisting}

	\null\textit{Пример}. Следующие выражения эквивалентны $True$:
	\begin{lstlisting}
	[1, 2, 3, 4, 5] !! 1 == 2
	[[]] !! 0 == []
	"Hello, world!" !! 1 == 'e'
	[[[1],[2],[3]]] !! 0 !! 1 == [2]
	\end{lstlisting}

	\null\textit{Реализация}, $O(n)$:
	\begin{lstlisting}
	getNth :: [a] -> Int -> a
	getNth [] _ = error "index too large"
	getNth (x:xs) 0 = x
	getNth (x:xs) n = getNth xs (n-1)
	\end{lstlisting}

	\null\textbf{last} -- равна последнему значению данного списка\\

	\textit{Тип}:
	\begin{lstlisting}
	last :: [a] -> a
	\end{lstlisting}

	\null\textit{Пример}. Следующие выражения эквивалентны $True$:
	\begin{lstlisting}
	last [1,2,3] == 3
	last "Hello!" == '!'
	\end{lstlisting}

	\null\textit{Реализация}, $O(n)$:
	\begin{lstlisting}
	last' :: [a] -> a
	last' [] = error "empty list"
	last' [x] = x
	last' (x:xs) = last' xs
	\end{lstlisting}

	\null\textbf{(++)} -- равна списку, составленному из двух данных посредством прямого слияния\\

	\textit{Тип}:
	\begin{lstlisting}
	(++) :: [a] -> [a] -> [a]
	\end{lstlisting}

	\null\textit{Пример}. Следующие выражения эквивалентны $True$:
	\begin{lstlisting}
	[1, 2, 3] ++ [4, 5, 6] == [1, 2, 3, 4, 5, 6]
	[1, 2, 3] ++ [] == [1, 2, 3]
	[] ++ [4, 5, 6] == [4, 5, 6]
	[] ++ [] == []
	"Hello," ++ " " ++ "world!" == "Hello, world!"
	\end{lstlisting}

	\null\textit{Плохая реализация}, $O(n^2)$:
	\begin{lstlisting}
	sumLists [] x = x
	sumLists x [] = x
	sumLists x y = sumLists (withoutLast x) (last x : y)

	withoutLast [x] = []
	withoutLast (x:xs) = x : withoutLast xs
	\end{lstlisting}

	\null\textit{Хорошая реализация}, $O(n_1)$:
	\begin{lstlisting}
	sumLists [] x = x
	sumLists x [] = x
	sumLists [x] (y:ys) = x : y : ys
	sumLists (x:xs) (y:ys) = x : sumLists xs (y:ys)
	\end{lstlisting}

	\null\textbf{length} -- равна длине данного списка\\

	\textit{Тип}:
	\begin{lstlisting}
	length :: [a] -> Int
	\end{lstlisting}

	\null\textit{Пример}. Следующие выражения эквивалентны $True$:
	\begin{lstlisting}
	length [1, 2, 3] == 3
	length "Hello" == 5
	length [] == 0
	\end{lstlisting}

	\null\textit{Реализация}, $O(n)$:
	\begin{lstlisting}
	length' [] = 0
	length' (x:xs) = 1+length'(xs)
	\end{lstlisting}

	\null\textbf{reverse} -- равна списку, полученному из данного посредством его обращения задом-наперед\\

	\textit{Тип}:
	\begin{lstlisting}
	reverse :: [a] -> [a]
	\end{lstlisting}

	\null\textit{Пример}. Следующие выражения эквивалентны $True$:
	\begin{lstlisting}
	length [1, 2, 3] == [3, 2, 1]
	length "Hello" == "olleH"
	length [] == []
	\end{lstlisting}

	\null\textit{Реализация}, $O(n)$:
	\begin{lstlisting}
	reverse' (x:xs) = reverse2 (x:xs) []
	reverse2 (x:xs) (a:as) = reverse2 xs (x:a:as)
	\end{lstlisting}

	\null\textbf{take} -- равна списку из первых $n$ значений данного списка\\

	\textit{Тип}:
	\begin{lstlisting}
	take :: Int -> [a] -> [a]
	\end{lstlisting}

	\null\textit{Пример}. Следующие выражения эквивалентны $True$:
	\begin{lstlisting}
	take 2 [1, 2, 3] == [1, 2]
	take 5 "Hello, world!" == "Hello"
	take 0 [5, 6, 7] == []
	take 5 [1, 2, 3] == [1, 2, 3]
	take (-2) [5, 6, 7] == []
	\end{lstlisting}

	\null\textit{Реализация}, $O(n)$:
	\begin{lstlisting}
	take' _ [] = []
	take' n (x:xs)
		| n<=0 = []
		| True = x : take' (n-1) xs
	\end{lstlisting}

	\null\textbf{drop} -- равна списку, полученному из данного удалением первых $n$ значений\\

	\textit{Тип}:
	\begin{lstlisting}
	drop :: Int -> [a] -> [a]
	\end{lstlisting}

	\null\textit{Пример}. Следующие выражения эквивалентны $True$:
	\begin{lstlisting}
	drop 2 [1, 2, 3] == [3]
	drop 5 "Hello, world!" == "Hello"
	drop 0 [5, 6, 7] == []
	drop 5 [1, 2, 3] == [1, 2, 3]
	drop (-2) [5, 6, 7] == []
	\end{lstlisting}

	\null\textit{Реализация}, $O(n)$:
	\begin{lstlisting}
	drop' _ [] = []
	drop' n (x:xs)
		| n<=0 = (x:xs)
		| True = drop' (n-1) xs
	\end{lstlisting}

	\null\textbf{filter} -- равна списку, полученному из данного посредством выбора значений, удовлетворяющих предикату\\

	\textit{Тип}:
	\begin{lstlisting}
	filter :: (a -> Bool) -> [a] -> [a]
	\end{lstlisting}

	\null\textit{Пример}. Следующие выражения эквивалентны $True$:
	\begin{lstlisting}
	filter (<4) [1, 4, 2, 65, 2, 1, 3, 5] == [1, 2, 2, 1, 3]
	filter (\t->False) "Oh shi..." == []
	\end{lstlisting}

	\null\textit{Реализация}, $O(n)$:
	\begin{lstlisting}
	filter' p (x:xs) = filter2 p (x:xs) []
	filter2 p (x:xs) (a:as)
		| p x = filter2 p xs (x:a:as)
		| True = filter2 p xs (a:as)
	\end{lstlisting}

	\null\textbf{map} -- равна списку, полученному из данного посредством применения данной функции к каждому значению этого списка\\

	\textit{Тип}:
	\begin{lstlisting}
	map :: (a -> b) -> [a] -> [b]
	\end{lstlisting}

	\null\textit{Пример}. Следующие выражения эквивалентны $True$:
	\begin{lstlisting}
	map (*2) [1, 2, 3, 4] == [2, 4, 6, 8]
	map (\t->sin t) [0, pi/2] == [0, 1]
	\end{lstlisting}

	\null\textit{Реализация}, $O(n)$:
	\begin{lstlisting}
	map' f [] = []
	map' f (x:xs) = (f x) : map' f xs
	\end{lstlisting}

\end{document}
