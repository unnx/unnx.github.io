\documentclass{article}
\usepackage[utf8]{inputenc}
\usepackage[russian]{babel}
\usepackage{cmap}
\usepackage{amsfonts,amsmath}
\usepackage{geometry}
\usepackage{fixint}
\usepackage{rumathgrk1}
\geometry{verbose,a4paper,tmargin=1cm,bmargin=1.5cm,lmargin=0.5cm,rmargin=0.5cm}
\pdfcompresslevel=9

\begin{document}
	
	\textbf{Плазма, дебаевский радиус экранирования, магнитное удержание}\\

	Плазмой называется частично или полностью ионизированный газ, в котором положительные и отрицательные заряды в среднем нейтрализуют друг друга. В общем случае плазма состоит из электронов, положительных ионов и нейтральных атомов (или молекул). Сила взаимодействия между атомами и молекулами в плазме уменьшается очень быстро, как $\frac{1}{r^7}$, поэтому взаимодействие между заряженными частицами играет большую роль. Вокруг каждого заряда преимущественно располагаются заряды противоположного знака, нейтрализующие действие данного заряда за пределами сферы радиуса $D$, называемого дебаевским радиусом экранировки. Его формула:
	\begin{equation}
		D = \sqrt{\varepsilon_0 kT/(ne^2)}
	\end{equation}

	($n$ -- концентрация электронов в плазме)\\

	Подстановка констант дает:
	\begin{equation}
		D = 69\sqrt{T/n}
	\end{equation}

	Внешнее электрическое поле проникает в плазму только на расстояния порядка дебаевского радиуса, т.е. плазма экранирует внешнее электрическое поле.\\

	Задачу удержания плазмы в ограниченном объеме нельзя решить, поместив ее в обычный сосуд, ибо стенки любого сосуда при такой температуре немедленно испарятся. Поэтому для удержания плазмы используются сильные магнитные поля.\\
	


\end{document}
