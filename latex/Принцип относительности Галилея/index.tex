\documentclass{article}
\usepackage[utf8]{inputenc}
\usepackage[russian]{babel}
\usepackage{cmap}
\usepackage{amsfonts,amsmath}
\usepackage{geometry}
\usepackage{fixint}
\usepackage{rumathgrk1}
\geometry{verbose,a4paper,tmargin=1cm,bmargin=1.5cm,lmargin=0.5cm,rmargin=0.5cm}
\pdfcompresslevel=9

\begin{document}
	
	\textbf{Принцип относительности Галилея}\\

	Для изучения механической системы необходимо выбрать систему отсчета. В разных системах отсчета время, вообще говоря, неоднородно, пространство неизотропно и неоднородно, свободное тело может не находиться в покое и т.д.\\

	Можно найти такую систему отсчета, в которой пространство является однородным и изотропным, а время -- однородным. Такая система называется инерциальной. В ней свободное тело может покоиться неограниченно долго.\\

	Однородность пространства и времени $\Rightarrow$ $L$ не зависит от $\vec r$ и $t$, т.е. является функцией скорости $\vec v$.\\

	Изотропность пространства $\Rightarrow$ $L$ не зависит от направления скорости, т.е. $L=L(\vec v^{\;2})=L(v^2)$.\\

	Таким образом, уравнения Лагранжа в инерциальной системе отсчета имеют вид:
	\begin{equation}
		\frac{d}{dt}\frac{\partial L}{\partial\vec v} = 0
	\end{equation}

	Отсюда следует, что $\vec v=const$.\\

	Получили закон инерции: в инерциальной системе отсчета всякое свободное движение происходит с постоянной по величине и направлению скоростью. Отсюда также следует, что если некоторая система движется с постоянной по величине и направлению скоростью относительно инерциальной системы, то она сама будет инерциальной. Таким образом, существует бесконечное множество инерциальных систем отсчета, движущихся друг относительно друга прямолинейно и равномерно. Во всех этих системах свойства пространства и времени одинаковы и одинаковы все законы механики. Это утверждение составляет содержание так называемого принципа относительности Галилея.\\

	Координаты $\vec r$ и $\vec r'$ одной и той же точки в двух различных инерциальных системах отсчета $K$ и $K'$, из которых вторая движется относительно первой со скоростью $\vec V$, связаны друг с другом соотношением:
	\begin{equation}
		\vec r = r' + Vt
	\end{equation}

	При этом считается, что ход времени одинаков в обеих системах отсчета:
	\begin{equation}
		t = t'
	\end{equation}

	Формулы $(2,3)$ называют преобразованиями Галилея. Принцип относительности Галилея можно сформулировать как требование инвариантности уравнений движения механики относительно этих преобразований.
\end{document}
