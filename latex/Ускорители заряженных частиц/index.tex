\documentclass{article}
\usepackage[utf8]{inputenc}
\usepackage[russian]{babel}
\usepackage{cmap}
\usepackage{amsfonts,amsmath}
\usepackage{geometry}
\usepackage{fixint}
\usepackage{rumathgrk1}
\geometry{verbose,a4paper,tmargin=1cm,bmargin=1.5cm,lmargin=0.5cm,rmargin=0.5cm}
\pdfcompresslevel=9

\begin{document}
	
	\textbf{Ускорители заряженных частиц}\\

	Генератор Ван де Графа\\

	Представляет собой металлический шар, называемый кондуктором, на изолирующей колонне. Внутри колонны движется замкнутая лента из шелка или прорезиненной ткани. Вблизи основания колонны установлена гребенка из остриев, подключенная к генератору напряжения. Заряды с остриев стекают на ленту, а затем переходят на внешнюю поверхность кондуктора. Лента также соединена с землей или электродом, на котором будет создаваться заряд другого знака. Если замкнуть кондуктор и этот электрод на т.н. разрядной трубке, в ней будут ускоряться заряды.\\

	Бетатрон\\

	Ускорение зарядов осуществляется с помощью вихревого электрического поля, создаваемого переменным магнитным. Состоит из тороидальной эвакуированной камеры, помещающейся между полюсами электромагнита специальной формы. Обмотка электромагнита питается переменным током с частотой около $100$ Гц. Переменное магнитное поле создает электрическое поле, разгоняющее заряженные частицы, а также удерживает их на орбите, совпадающей с осью камеры. Чтобы электрон не вылетал с орбиты, индукция поля на орбите $B_\text{орб}$ должна быть связана с индукцией поля $B$ внутри нее соотношением:
	\begin{equation}
		B_\text{орб} = \frac{1}{2}<B>
	\end{equation}

	После разгона электроны иногда перенаправляют в мишень. При столкновении испускается жесткое гамма-излучение.\\

	Циклотрон\\

	Частицы втягиваются в две полукруглые коробки (дуанты), помещенные в откачиваемый корпус и пересекаемые перпендикулярным их плоскости магнитным полем. На дуанты подается переменное напряжение, так, что его знак меняется при прохождении частицей границы между дуантами. Магнитное поле заставляет частицы двигаться по окружности. Чем больше скорость частиц, тем больше радиус такой окружности. Следовательно, частицы движутся по траектории, близкой к спирали. При больших скоростях сказывается зависимость периода обращения частицы от ее скорости, поэтому переменное напряжение с неизменной частотой перестает их ускорять. Ускорители, где это учтено, и частота изменения напряжения меняется со временем, назиываются синхроциклотронами, или фазотронами. Ускорители, где вместо частоты напряжения меняется индукция магнитного поля, называются синхротронами. Ускоритель, в котором изменяется и частота напряжения, и индукция магнитного поля, называется синхрофазотроном. В нем частицы движутся не по спирали, а по окружности.



\end{document}
