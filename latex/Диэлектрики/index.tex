\documentclass{article}
\usepackage[utf8]{inputenc}
\usepackage[russian]{babel}
\usepackage{cmap}
\usepackage{amsfonts,amsmath}
\usepackage{geometry}
\usepackage{fixint}
\usepackage{rumathgrk1}
\geometry{verbose,a4paper,tmargin=1cm,bmargin=1.5cm,lmargin=0.5cm,rmargin=0.5cm}
\pdfcompresslevel=9

\begin{document}
	
	\textbf{Диэлектрики}\\

	Диэлектриками являются вещества с ничтожно малым количеством свободных носителей заряда, т.е. почти не проводящие ток. \\

	Каждая молекула вещества содержит некоторое количество зарядов. У некоторых молекул эти заряды разнесены на некоторое расстояние друг от друга. Такую молекулу можно считать диполем с моментом $\vec p$. Молекулы такого рода называют полярными. У неполярных молекул -- наоборот, заряды в среднем сконцентрированы в одной точке ($\vec p = 0$).\\

	Диэлектрик как с полярными, так и с неполярными молекулами взаимодействует с внешним электрическим полем. Полярные молекулы, являясь диполями, стремятся повернуться в направлении поля (им мешает тепловое движение). Неполярные же молекулы поляризуются: их заряды под действием поля расходятся на некоторое расстояние и у молекулы появляется ненулевой дипольный момент $\vec p$. Дипольный момент молекулы равен:
	\begin{equation}
		\vec p = \sum_i q_i<\vec r_i>
	\end{equation}

	Требуется связать дипольный момент молекулы с напряженностью поля. Для этого вводится коэффициент пропорциональности $\beta\varepsilon_0$:
	\begin{equation}
		\vec p = \beta\varepsilon_0\vec E
	\end{equation}

	Величина $\beta$ называется поляризуемостью молекулы.\\

	Дипольный момент молекулы -- микрохарактеристика. Чтобы охарактеризовать диэлектрик в целом, нужно ввести что-то вроде "плотности дипольного момента". Такая величина носит название поляризованности диэлектрика:
	\begin{equation}
		\vec P = \frac{1}{\Delta V}\sum_{\Delta V}\vec p
	\end{equation}

	Вообще говоря, поляризованность может быть непостоянной -- зависеть от точки внутри диэлектрика, взятого объема $\Delta V$. Если она зависит только от величины взятого объема, диэлектрик называется изотропным.\\

	Для изотропных диэлектриков можно ввести характеристику, связывающую поляризованность с напряженностью электрического поля. Она называется диэлектрической восприимчивостью диэлектрика $\kappa$:
	\begin{equation}
		\vec P = \kappa\varepsilon_0\vec E
	\end{equation}

	Сопоставим выражения для $\beta$ и $\kappa$:
	\begin{equation}
		\sum_{\Delta V}\vec p = n\Delta V \beta \varepsilon_0 \vec E
	\end{equation}
	\begin{equation}
		\vec P = n \beta \varepsilon_0 \vec E
	\end{equation}
	\begin{equation}
		\kappa = n \beta
	\end{equation}

	Связали макро- и микрохарактристику поляризации диэлектрика через концентрацию молекул.\\

	Выражения для $\beta$ и $\kappa$ верны как в случае полярных, так и в случае неполярных молекул. \\

	Молекулы диэлектрика изменяют электрическое поле, в которое он помещен, потому что, имея дипольный момент, создают собственное поле. Таким образом, результирующее поле складывается из двух:
	\begin{equation}
		\vec E = \vec E_0 + \vec E'
	\end{equation}

	(Поле в отсутствие диэлектрика обозначим через $\vec E_0$, поле, создаваемое диэлектриком -- через $\vec E'$)\\

	Заряды, создающие поле $\vec E_0$ называются сторонними или свободными, а те, которые создают $\vec E'$ -- связанными.\\

	В объеме диэлектрика молекулы-диполи выстраиваются так, что почти не действуют на внешнее поле (ибо отрицательные заряды стремятся расположиться близко к положительным -- нейтрализовать их). Отсюда следует, что поле $E'$ создают связанные заряды, выступившие на поверхность диэлектрика.\\

	Найдем заряд, выступающий на поверхность диэлектрика. Для этого представим диэлектрическую стенку некоторой толщины. Вектор $\vec P$ имеет в ней постоянное направление. Построим косой цилиндр с основаниями площади $\Delta S$, лежащими на двух поверхностях нашей стенки. Пусть образующие цилиндра параллельны вектору $\vec P$. Объем цилиндра:
	\begin{equation}
		\Delta V = l\Delta S \cos\alpha
	\end{equation}

	Здесь $l$ -- длина образующей, а $\cos\alpha$ -- угол между $\vec P$ и нормалью к стенке $\vec n$.\\

	Чтобы найти суммарный дипольный момент молекул, заключенных в объеме, умножим его на $P$:
	\begin{equation}
		P\Delta V = Pl\Delta S \cos\alpha
	\end{equation}

	Наш цилиндр представляет собой диполь с зарядами $+|q|=+|\sigma'|\Delta S$ и $-|q|=-|\sigma'|\Delta S$, где $|\sigma'|$ -- модуль поверхностной плотности связанного заряда. Проекция дипольного момента цилиндра на $\vec P$ равна заряду, помноженному на расстояние между зарядами:
	\begin{equation}
		Pl\Delta S\cos\alpha = ql = \sigma'\Delta S l
	\end{equation}

	Отсюда:
	\begin{equation}
		\sigma' = P\cos\alpha = P_n
	\end{equation}

	Расписав $P$, получим:
	\begin{equation}
		\sigma' = \kappa\varepsilon_0 E_n
	\end{equation}

	На противоположные поверхности нашей стенки под действием внешнего поля выступают заряды с плотностями $+|\sigma'|$ и $|-\sigma'|$. Ясно, что вектор $\vec P$ при этом стремится направиться по линиям напряженности внешнего поля.\\

	При включении внешнего поля полярные молекулы стремятся повернуться по полю. Вследствие этого, объемная плотность связанных зарядов $\rho'$ постоянна (т.к. в среднем постоянно расстояние между зарядами). Если внутри такого диэлектрика зафиксировать поверхность, при включении внешнего поля заряд через нее не потечет. Другое дело -- диэлектрики с неполярными молекулами.\\

	Найдем объемную плотность связанных зарядов на примере диэлектриков с неполярными молекулами. Для этого внутри диэлектрика выберем косой цилиндр с параллельными $\vec P$ образующими длиной $l_1+l_2$, пересекающий плоскость, параллельную основаниям площади $\Delta S$. Этот цилиндр выберем так, что при включении внешнего поля плоскость внутри него пересекают разноименные заряды из двух частей цилиндра (с образующими $l_1$ и $l_2$). Ясно, что эти заряды движутся в разные стороны. Таким образом, плоскость пересекут все заряды, находящиеся в двух объемах: $l_1\Delta S\cos\alpha$ и $l_2\Delta S\cos\alpha$ ($\Delta S$ -- площадь основания цилиндра, $\alpha$ -- угол между $\vec P$ и нормалью к поверхности цилиндра $\vec n$). Таким образом, после включения поля через плоскость внутри цилиндра пройдет число зарядов $\Delta q'$:
	\begin{equation}
		\Delta q' = |e|nl_1\Delta S\cos\alpha + |e|nl_2\Delta S\cos\alpha = |e|n(l_1+l_2)\Delta S\cos\alpha
	\end{equation}

	Здесь $n$ -- плотность зарядов.

	Заряды переносятся попарно: перенос отрицательного заряда в одну сторону сопровождается переносом положительного заряда в другую. Каждую такую пару можно считать диполем. Число пар в единице объема равно $n$. Дипольный момент каждой пары равен $e(l_1+l_2)$, а выражение $e(l_1+l_2)n$ эквивалентно $pn$, т.е. модулю поляризованности $P$. Таким образом, получаем:
	\begin{equation}
		\Delta q' = P\Delta S \cos\alpha = P_n\Delta S = (\vec P, \vec S)
	\end{equation}

	Перейдем от дельт к дифференциалам:
	\begin{equation}
		dq' = (\vec P;d\vec S)
	\end{equation}

	Отсюда мгновенно следует выражение объемного связанного заряда через поток вектора $\vec P$:
	\begin{equation}
		q'_{out} = \oint_S dq' = \oint_S \vec Pd\vec S
	\end{equation}

	Индекс $out$ означает, что $q_{out}$ -- заряд, вышедший из объема, ограниченного замкнутой поверхностью $S$. Само собой, он равен вошедшему заряду с обратным знаком:
	\begin{equation}
		q'_{in} = -q_{out} = -\oint_S \vec Pd\vec S
	\end{equation}

	Наконец, перейдем к объемной плотности зарядов:
	\begin{equation}
		q'_{in} = \int_V \rho' dV
	\end{equation}

	Получаем:
	\begin{equation}
		\int_V \rho'dV = -\oint_S \vec P d\vec S
	\end{equation}

	(Поверхность $S$ ограничивает объем $V$)\\

	По теореме Остроградского-Гаусса:
	\begin{equation}
		\int_V \rho'dV = -\int_V(\vec\nabla;\vec P)dV
	\end{equation}
	\begin{equation}
		\rho' = -(\vec\nabla;\vec P)
	\end{equation}

	Получили выражение для объемной плотности связанных зарядов в диэлектрике с неполярными молекулами. Величина $\kappa$ для разных веществ выбирается таким образом, чтобы эта формула была верна и для диэлектриков с полярными молекулами.



\end{document}

