\documentclass{article}
\usepackage[utf8]{inputenc}
\usepackage[russian]{babel}
\usepackage{cmap}
\usepackage{amsfonts,amsmath}
\usepackage{geometry}
\usepackage{fixint}
\usepackage{rumathgrk1}
\geometry{verbose,a4paper,tmargin=1cm,bmargin=1.5cm,lmargin=0.5cm,rmargin=0.5cm}
\pdfcompresslevel=9

\begin{document}
	
	\textbf{Закон Ома}\\

	Если внутри проводника создать электрическое поле, по нему потечет ток, и заряды переместятся так, чтобы поле исчезло. Поэтому для поддержания этого поля, и, соответственно, тока, требуется наличие некоторых сторонних сил, перемещающих заряды. Это силы неэлектростатического происхождения, но их можно охарактеризовать напряженностью поля сторонних сил $\vec E^*$. Работа сторонних сил, очевидно, зависит от количества зарядов, над которыми она совершается.\\

	Вводится величина, характеризующая удельную работу сторонних сил:
	\begin{equation}
		\mathcal{E} = \frac{A}{q}
	\end{equation}

	Распишем работу сторонних сил на участке цепи $1$ -- $2$:
	\begin{equation}
		A_{12} = \int_1^2\vec F^* d\vec l = q\int_1^2 \vec E^* d\vec l
	\end{equation}

	Отсюда:
	\begin{equation}
		\mathcal{E}_{12} = \int_1^2 \vec E^* d\vec l
	\end{equation}

	Ом экспериментально установил, что в любом проводнике при отсутствии сторонних сил сила тока пропорциональна напряжению:
	\begin{equation}
		I = \frac{1}{R} U
	\end{equation}

	Величина $R$, входящая в коэффициент пропорциональности $\frac{1}{R}$, называется сопротивлением проводника.\\

	Величина сопротивления зависит от формы и размеров проводника:
	\begin{equation}
		R = \rho\frac{l}{S}
	\end{equation}

	$\rho$ -- удельное сопротивление. \\

	Если сторонние силы совершают работу $q\mathcal{E}$, то закон Ома имеет вид:
	\begin{equation}
		I = \frac{\varphi_1-\varphi_2 + \mathcal{E}}{R}
	\end{equation}

	Для замкнутой цепи $\varphi_1 = \varphi_2$, поэтому
	\begin{equation}
		I = \frac{\mathcal{E}}{R}
	\end{equation}

	Кирхгоф обобщил закон Ома для электрических цепей и вывел два правила:\\
	1) Сумма токов, сходящихся в узле, равна нулю. При этом токи, приходящие в узел, и выходящие из него, берутся с разными знаками.\\
	2) Для любого замкнутого контура, выбранного в цепи, выполняется:
	\begin{equation}
		\sum I_kR_k = \sum \mathcal{E}_k
	\end{equation}
\end{document}
