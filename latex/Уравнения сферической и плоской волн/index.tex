\documentclass{article}
\usepackage[utf8]{inputenc}
\usepackage[russian]{babel}
\usepackage{cmap}
\usepackage{amsfonts,amsmath}
\usepackage{geometry}
\usepackage{fixint}
\usepackage{rumathgrk1}
\geometry{verbose,a4paper,tmargin=1cm,bmargin=1.5cm,lmargin=0.5cm,rmargin=0.5cm}
\pdfcompresslevel=9

\begin{document}
	
	\textbf{Уравнения сферической и плоской волн}\\

	1) Сферическая волна\\

	Сферическая продольная волна одномерна: ее волновая поверхность -- сфера, поэтому смещение $\xi$ зависит только от одной координаты $r$:
	\begin{equation}
		\xi = \xi(r,t)
	\end{equation}

	Решение волнового уравнения для одномерной упругой волны имеет вид:
	\begin{equation}
		\xi(x,t) = a\cos(\omega t - \frac{\omega}{v}x + \varphi)
	\end{equation}

	Если энергия волны не поглащается средой, то $a=const$, равная амплитуде на единичном расстоянии от источника волны. В случае сферического распространения, энергия волны с увеличением $r$ приходится на все большую волновую поверхность. Таким образом, в уравнение сферической волны вносится поправка:
	\begin{equation}
		\xi(r,t) = \frac{a}{r}\cos(\omega t - \frac{\omega}{v}x + \varphi)
	\end{equation}

	(Причина появления коэффициента $\frac{1}{r}$ в том, что в выражении для энергии волны, которая должна быть постоянной при отсутствии поглащения средой, амплитуда и радиус стоят во второй степени: $a^2r^2=const$. См. энергия упругих волн)\\

	2) Плоская волна\\

	Пусть колебания в плоскости, проходящей через начало координат, имеют вид:
	\begin{equation}
		\xi_0 = a\cos(\omega t + \varphi)
	\end{equation}

	Рассмотрим волновую поверхность (плоскость), отстоящую от данной на расстояние $l$. Колебания в этой плоскости будут отставать на время $\tau = \frac{l}{v}$:
	\begin{equation}
		\xi_l = \xi = a\cos(\omega(t-\frac{l}{v})+\varphi) = a\cos(\omega t - kl + \varphi)
	\end{equation}

	Выразим $l$ через радиус-вектор $\vec r$ -- тогда мы сможем найти колебания волны в любой точке пространства:
	\begin{equation}
		(\vec n; \vec r) = l
	\end{equation}

	$\vec n$ -- нормаль, соответствующая направлению распространения волны. Отсюда:
	\begin{equation}
		\xi = a\cos(\omega t - (k\vec n;\vec r) + \varphi)
	\end{equation}

	Введем вектор $\vec k = k\vec n$ (он называется волновым вектором). Таким образом, уравнение представимо в виде:
	\begin{equation}
		\xi = a\cos(\omega t - (\vec k;\vec r) + \varphi)
	\end{equation}

	Можно разложить скалярное произведение $(\vec k;\vec r)$ по координатам и получить:
	\begin{equation}
		\xi(x,y,z,t) = a\cos(\omega t - k_xx - k_yy - k_zz + \varphi)
	\end{equation}

	Поскольку $(\vec k;\vec r) = \frac{2\pi}{\lambda}(\vec n;\vec r)$:
	\begin{equation}
		k_x = \frac{2\pi}{\lambda}\cos\alpha,\;\;k_y= \frac{2\pi}{\lambda}\cos\beta,\;\;k_z = \frac{2\pi}{\lambda}\cos\gamma
	\end{equation}

	$\alpha, \beta, \gamma$ -- углы, между направлением распространения волны и осями $x, y, z$ соответственно.
\end{document}
