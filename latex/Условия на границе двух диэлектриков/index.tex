\documentclass{article}
\usepackage[utf8]{inputenc}
\usepackage[russian]{babel}
\usepackage{cmap}
\usepackage{amsfonts,amsmath}
\usepackage{geometry}
\usepackage{fixint}
\usepackage{rumathgrk1}
\geometry{verbose,a4paper,tmargin=1cm,bmargin=1.5cm,lmargin=0.5cm,rmargin=0.5cm}
\pdfcompresslevel=9

\begin{document}
	
	\textbf{Условия на границе двух диэлектриков}\\

	Известно:
	\begin{equation}
		[\vec\nabla;\vec E] = 0,\;\;\;(\vec\nabla;\vec D) = 0
	\end{equation}

	Отсюда можем выяснить, как изменяются векторы $\vec E$ и $\vec D$ на границе двух диэлектриков, как преломляются их линии.\\

	Вблизи границы выберем прямоугольный контур со сторонами $l$ и $h$ (сторона $h$ перпендикулярна границе и ее длина стремится к нулю). По теореме Стокса циркуляция $\vec E$ по этому контуру должна быть равна нулю, т.к. он замкнутый:
	\begin{equation}
		[\vec\nabla;\vec E]=0 \Rightarrow \oint_l \vec E d\vec l = 0
	\end{equation}

	Относительно границы раздела можно представить вектор $\vec E$ через нормальную составляющую $E_n$ и тангенциальную $E_\tau$. Проходя через границу раздела, эти составляющие изменяются. Возьмем криволинейный интеграл:
	\begin{equation}
		\oint \vec E d\vec l = E_{\tau1}l- E_{\tau2}l = 0
	\end{equation}

	Получили соотношение для тангенциальной составляющей $\vec E$:
	\begin{equation}
		E_{\tau1} = E_{\tau2}
	\end{equation}

	(Мы пренебрегаем значением интеграла по частям контура длины $h$, т.к. эта длина стремится к нулю)\\

	Теперь выберем поверхность параллелепипеда, пересекающую границу (основания площади $S$ и высота $h\rightarrow 0$).\\

	По теореме Остроградского-Гаусса выражаем поток $\vec D$ через эту поверхность:
	\begin{equation}
		\oint_S \vec D d\vec S = 0
	\end{equation}

	Берем интеграл, пренебрегая потоком через боковые стенки высоты $h$:
	\begin{equation}
		\oint_S \vec D d\vec S = D_{n1}S - D_{n2} = 0
	\end{equation}

	При вычислении потока мы всюду спроектировали $\vec S$ на одну нормаль, поэтому получили разные знаки у слагаемых.\\

	Таким образом:
	\begin{equation}
		D_{n1}S = D_{n2}
	\end{equation}

	Выяснили, как изменяется тангенциальная составляющая $\vec E$ и нормальная составляющая $\vec D$ при проходе через границу раздела диэлектриков.\\

	Чтобы узнать, как изменяются $E_n$ и $D_\tau$ на границе, раскроем $D$ через $E$ (и наоборот):
	\begin{equation}
		D_{n1} = D_{n2}
	\end{equation}
	\begin{equation}
		\varepsilon_0\varepsilon_1 E_{n1} = \varepsilon_0\varepsilon_2 E_{n2}
	\end{equation}
	\begin{equation}
		\frac{E_{n1}}{E_{n2}} = \frac{\varepsilon_2}{\varepsilon_1}
	\end{equation}

	Аналогично найдем изменение $D_\tau$:
	\begin{equation}
		E_{\tau1} = E_{\tau2}
	\end{equation}
	\begin{equation}
		\frac{D_{\tau1}}{\varepsilon_0\varepsilon_1} = \frac{D_{\tau2}}{\varepsilon_0\varepsilon_2}
	\end{equation}
	\begin{equation}
		\frac{D_{\tau1}}{D_{\tau2}} = \frac{\varepsilon_1}{\varepsilon_2}
	\end{equation}

	Выяснили, что на границе двух диэлектриков происходит следующее:
	\begin{equation}
		E_{\tau1} = E_{\tau2},\;\;\; \frac{E_{n1}}{E_{n2}} = \frac{\varepsilon_2}{\varepsilon_1}
	\end{equation}
	\begin{equation}
		D_{n1} = D_{n2},\;\;\; \frac{D_{\tau1}}{D_{\tau2}} = \frac{\varepsilon_1}{\varepsilon_2}
	\end{equation}

\end{document}
