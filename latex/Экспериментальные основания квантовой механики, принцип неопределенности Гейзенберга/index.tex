\documentclass{article}
\usepackage[utf8]{inputenc}
\usepackage[russian]{babel}
\usepackage{cmap}
\usepackage{amsfonts,amsmath}
\usepackage{geometry}
\usepackage{fixint}
\usepackage{rumathgrk1}
\geometry{verbose,a4paper,tmargin=1cm,bmargin=1.5cm,lmargin=0.5cm,rmargin=0.5cm}
\pdfcompresslevel=9

\begin{document}

	\textbf{Экспериментальные основания квантовой механики, принцип неопределенности Гейзенберга}\\

	Луи де Бройль выдвинул гипотезу: движение любой микрочастицы связано с волновым процессом, длина волны которого равна $\lambda=\frac{2\pi\hbar}{p}=\frac{2\pi\hbar}{mv}$, а частота $\omega=\frac{E}{\hbar}$. Вскоре она была подтверждена экспериментально.\\

	Таким образом, микрочастица проявляет волновые свойства, а некоторые волновые процессы -- корпускулярные.\\

	В классической механике состояние материальной точки определяется заданием значений координат, импульса, энергии и т.д. Эти величины называются динамическими переменными. На самом деле, микрообъекту не могут быть приписаны указанные динамические переменные, т.к. имеет место соотношение неопределенности Гейзенберга:
	\begin{equation}
		\Delta A \cdot \Delta B \geq \frac{\hbar}{2}
	\end{equation}

	$\Delta A, \Delta B$ -- величины, называемые канонически сопряженными.\\

	Канонически сопряженными являются энергия и время:
	\begin{equation}
		\Delta E\Delta t \geq \frac{\hbar}{2}
	\end{equation}

	Проекция импульса и соответствующая координата:
	\begin{equation}
		\Delta p_x \Delta x \geq \frac{\hbar}{2}
	\end{equation}

	Энергия электрона в атоме водорода равна:
	\begin{equation}
		E=\frac{p^2}{2m}-\frac{e^2}{r}
	\end{equation}

	Заменим $p$ на $\hbar/r$ по принципу неопределенности:
	\begin{equation}
		E = \frac{\hbar^2}{2mr^2} - \frac{e^2}{r}
	\end{equation}

	Найдем минимальные возможные радиус и энергию. Для этого продифференцируем предыдущее выражение по $r$ и приравняем его к нулю:
	\begin{equation}
		-\frac{\hbar^2}{mr^3}+\frac{e^2}{r^2}=0
	\end{equation}

	Решая это уравнение, получим:
	\begin{equation}
		r_{min} = \frac{\hbar^2}{(me^2)}
	\end{equation}

	(Данное выражение совпадает с первой боровской орбитой).\\

	Подстановка в формулу для энергии дает минимальную энергию:
	\begin{equation}
		E_{min} = \frac{\hbar^2}{2m}(\frac{me^2}{\hbar^2})^2 - e^2\frac{me^2}{\hbar^2} = -\frac{me^4}{2\hbar^2}
	\end{equation}

	
	
\end{document}
