\documentclass{article}
\usepackage[utf8]{inputenc}
\usepackage[russian]{babel}
\usepackage{cmap}
\usepackage{amsfonts,amsmath}
\usepackage{geometry}
\usepackage{fixint}
\usepackage{rumathgrk1}
\geometry{verbose,a4paper,tmargin=1cm,bmargin=1.5cm,lmargin=0.5cm,rmargin=0.5cm}
\pdfcompresslevel=9

\begin{document}

	\textbf{Энергетические уровни молекулы в гармоническом приближении}\\

	Энергия молекулы складывается из трех частей:
	\begin{equation}
		E = E_e + E_v + E_r
	\end{equation}

	$E_e$ -- энергия, обусловленная электронной конфигурацией\\

	$E_v$ -- энергия, соответствующая колебаниям молекулы\\

	$E_r$ -- энергия, соответствующая вращениям молекулы\\

	Энергия квантового осциллятора:
	\begin{equation}
		E_v = (v+1/2)\hbar\omega_v\;\;\;v=0,1,2,...
	\end{equation}

	$v$ -- колебательное квантовое число с правилом отбора $\Delta v = \pm 1$.\\

	Энергия системы с моментом инерции $I$ и вращающейся с угловой скоростью $\omega_r$ равна
	\begin{equation}
		E_r = \frac{I\omega_r^2}{2}=\frac{(I\omega_r)^2}{2I}=\frac{M^2}{2I}
	\end{equation}

	Момент импульса может принимать лишь дискретные значения:
	\begin{equation}
		M=\hbar\sqrt{J(J+1)}\;\;\;J=0,1,2,...
	\end{equation}

	Поэтому вращательная энергия молекулы может принимать значения:
	\begin{equation}
		E_r = \frac{\hbar^2 J(J+1)}{2I}
	\end{equation}

	Для вращательного числа есть правило отбора $\Delta J=\pm 1$
	
\end{document}
