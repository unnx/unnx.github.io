\documentclass{article}
\usepackage[utf8]{inputenc}
\usepackage[russian]{babel}
\usepackage{cmap}
\usepackage{amsfonts,amsmath}
\usepackage{geometry}
\usepackage{fixint}
\usepackage{rumathgrk1}
\geometry{verbose,a4paper,tmargin=1cm,bmargin=1.5cm,lmargin=0.5cm,rmargin=0.5cm}
\pdfcompresslevel=9

\begin{document}

	\textbf{Принцип Паули, рентгеновские спектры}\\

	Состояние каждого электрона в атоме характеризуется четырьмя квантовыми числами:\\

	главным -- $n,\;\;\;(n=1,2,3,...)$	\\

	азимутальным -- $l,\;\;\;(l=0,1,2,...)$	\\

	магнитным -- $m_l,\;\;\;(m_l=-l,...,0,...,l)$ \\

	спиновым -- $m_s,\;\;\;(m_s=\pm 1/2)$ \\

	Принцип Паули гласит: в одном и том же атоме не может быть двух электронов с одинаковым набором квантовых чисел.\\

	При бомбардировке электронами вещества помимо тормозного излучения появляется характеристическое. Спектры характеристического излучения состоят из серий, обозначаемых $K$, $L$, $M$, $N$, $O$, ..., каждая серия насчитывает небольшое число линий, обозначаемых буквами $\alpha, \beta, \gamma,...$.\\

	Частоты линий рентгеновских спектров выражается законом Мозли:
	\begin{equation}
		\omega = R[\frac{(Z-\sigma_1)}{n_1^2}-\frac{(Z-\sigma_2)^2}{n_2^2}]
	\end{equation}

	$\sigma_1, \sigma_2$ называются постоянными экранирования.


\end{document}
