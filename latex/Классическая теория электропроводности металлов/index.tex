\documentclass{article}
\usepackage[utf8]{inputenc}
\usepackage[russian]{babel}
\usepackage{cmap}
\usepackage{amsfonts,amsmath}
\usepackage{geometry}
\usepackage{fixint}
\usepackage{rumathgrk1}
\geometry{verbose,a4paper,tmargin=1cm,bmargin=1.5cm,lmargin=0.5cm,rmargin=0.5cm}
\pdfcompresslevel=9

\begin{document}
	
	\textbf{Классическая теория электропроводности металлов}\\

	Исходя их представлений о свободных электронах Друде создал классическую теорию электропроводности металлов, которая затем была усовершенствована Лоренцем.\\

	Друде предположил, что электроны проводимости в металле ведут себя подобно молекулам идеального газа. У них есть средняя длина свободного пробега $\lambda$ и они сталкиваются с ионами, образующими кристаллическую решетку. Среднюю скорость теплового движения электронов можно оценить:
	\begin{equation}
		<v>=\sqrt{8kT/\pi m}
	\end{equation}

	Закон Ома\\

	Друде считал, что при соударении электрона с ионом кристаллической решетки приобретенная электроном за время свободного пробега дополнительная энергия передается иону. Предположим, что поле, ускоряющее электроны, однородно. Тогда под действием поля электрон получит постоянное ускорение, равное $eE/m$ и к концу свободного пробега скорость упорядоченного движения достигнет в среднем значения:
	\begin{equation}
		u_{max} = \frac{eE}{m}\tau
	\end{equation}

	Здесь $\tau$ -- время между двумя последовательными соударениями электрона и ионов.\\

	Друде не учитывал распределение электронов по скоростям и приписывал всем электронам одинаковое значение скорости $v$, поэтому
	\begin{equation}
		\tau = \frac{\lambda}{v}
	\end{equation}

	В итоге:
	\begin{equation}
		u_{max} = \frac{eE\lambda}{mv}
	\end{equation}

	Среднее значение $<u>$ за пробег равно $\frac{1}{2}u_{max}$ (т.к. $u$ изменяется линейно).\\

	Поскольку $j = ne<u>$, получаем:
	\begin{equation}
		j = \frac{ne^2\lambda}{2mv}E
	\end{equation}

	Плотность тока пропорциональна напряженности поля. Мы пришли к закону Ома и нашли выражение для удельной проводимости:
	\begin{equation}
		\sigma = \frac{ne^2\lambda}{2mv}
	\end{equation}

	Закон Джоуля-Ленца\\

	Друде полагал, что среднее приращение кинетической энергии электронов за счет их свободного движения равно:
	\begin{equation}
		<\Delta\varepsilon_k> = \frac{m<u^2>}{2}
	\end{equation}

	Подставим выражение для $u$, найдя среднюю кинетическую энергию в конце свободного пробега:
	\begin{equation}
		<\Delta\varepsilon_{k_{max}}> = \frac{me^2\lambda^2E^2}{2m^2v^2}
	\end{equation}

	Удельная тепловая мощность, выделяемая за счет соударений электронов и ионов кристаллической решетки, будет равна произведению числа электронов в единице объема на число соударений в единицу времени и на найденную энергию, передаваемую во время соударений:
	\begin{equation}
		Q_{\text{уд}} = n\frac{1}{\tau}<\Delta\varepsilon_{k_{max}}> = n\frac{v}{\lambda}<\Delta\varepsilon_{k_{max}}>
	\end{equation}
	\begin{equation}
		Q_{\text{уд}} = \frac{ne^2\lambda}{2mv}E^2
	\end{equation}
	\begin{equation}
		Q_{\text{уд}} = \sigma E^2
	\end{equation}

	Поскольку из закона Ома $j^2 = \sigma^2 E^2$ или $j^2 = \frac{1}{\rho}\sigma E^2$, выходит, что:
	\begin{equation}
		Q_{\text{уд}} = \rho j^2
	\end{equation}

	Это выражение совпадает с законом Джоуля-Ленца, записанном в дифференциальной форме.

\end{document}
