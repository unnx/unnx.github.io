\documentclass{article}
\usepackage[utf8]{inputenc}
\usepackage[russian]{babel}
\usepackage{cmap}
\usepackage{amsfonts,amsmath}
\usepackage{geometry}
\usepackage{fixint}
\usepackage{rumathgrk1}
\geometry{verbose,a4paper,tmargin=1cm,bmargin=1.5cm,lmargin=0.5cm,rmargin=0.5cm}
\pdfcompresslevel=9

\begin{document}
	
	\textbf{Линейные ОДУ 1-го порядка}\\

	Линейным ОДУ первого порядка является уравнение:
	\begin{equation}
		y' + a(x)y + b(x) = 0
	\end{equation}

	Если $b(x)$ тождественно $0$, то уравнение $(1)$ называется однородным, в противном случае -- неоднородным линейным ОДУ 1-го порядка.\\

	Однородное линейное ОДУ 1-го порядка является ОДУ с разделяющимися переменными:
	\begin{equation}
		y' = -a(x)y
	\end{equation}

	Для решения линейных ОДУ 1-го порядка применяют метод Лагранжа вариации произвольных постоянных. Сначала находят общее решение однородного уравнения:
	\begin{equation}
		y = y(x)+C
	\end{equation}

	Затем произвольную постоянную обозначают как $C(x)$ и подставляют $y$ и $y'$ в исходное уравнение, находя $C(x)$. Затем ответ записывают в виде:
	\begin{equation}
		y(x) + C(x)
	\end{equation}
\end{document}
