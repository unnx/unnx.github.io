\documentclass{article}
\usepackage[utf8]{inputenc}
\usepackage[russian]{babel}
\usepackage{cmap}
\usepackage{amsfonts,amsmath}
\usepackage{geometry}
\usepackage{fixint}
\usepackage{rumathgrk1}
\geometry{verbose,a4paper,tmargin=1cm,bmargin=1.5cm,lmargin=0.5cm,rmargin=0.5cm}
\pdfcompresslevel=9

\begin{document}
	
	\textbf{Излучение диполя}\\

	Простейшей системой, излучающей электромагнитные волны, является колеблющийся электрический диполь, например, неподвижный точечный заряд $+q$ и колеблющийся около него точечный заряд $-q$. Дипольный момент такой системы изменяется по закону:
	\begin{equation}
		\vec p = -q\vec r = -q\vec l\cos\omega t = \vec p_m\cos\omega t
	\end{equation}

	($\vec l$ -- вектор оси диполя, $\vec r$ -- радиус-вектор отрицательного заряда)\\

	Рассмотрим излучение диполя, размеры которого малы по сравнению с длиной волны ($l << \lambda$). Такой диполь называется элементарным. В непосредственной близости от диполя картина электромагнитного поля очень сложна. Она сильно упрощается в так называемой волновой зоне диполя, которая начинается на расстояниях $r$, много больших длины волны.\\

	Если волна распространяется в однородной изотропной среде, то волновой фронт в волновой зоне будет сферическим.\\

	Амплитуды $E_m$ и $H_m$ излучаемой волны зависят от расстояния до диполя $r$, а также от угла $\vartheta$, отсчитываемого от луча, перпендикулярного оси диполя. Зависимость для вакуума имеет вид:
	\begin{equation}
		E_m \sim H_m \sim \frac{1}{r}\sin\vartheta
	\end{equation}

	Среднее значение плотности потока энергии $<S>$ пропорционально произведению $E_mH_m$, поэтому:
	\begin{equation}
		<S>\sim\frac{1}{r^2}\sin^2\vartheta
	\end{equation}

	-- сильнее всего диполь излучает в направлении, перпендикулярном его оси.\\

	Мощность излучения диполя пропорциональна квадрату второй производной дипольного момента по времени:
	\begin{equation}
		P\sim (\frac{\partial^2\vec p}{\partial t^2})^2
	\end{equation}

	Поскольку $(\frac{\partial^2\vec p}{\partial t^2})^2 = p_m^2\omega^4\cos^2\omega t$:
	\begin{equation}
		P\sim p_m^2\omega^4\cos^2\omega t
	\end{equation}

	Усредним это выражение по времени:
	\begin{equation}
		<P>\sim p_m^2\omega^4
	\end{equation}

	С другой стороны, $\frac{\partial^2\vec p}{\partial t^2}=-q\vec a$, где $\vec a$ -- ускорение колеблющегося заряда. Подставим в предыдущую формулу:
	\begin{equation}
		P\sim q^2a^2
	\end{equation}

	В СИ коэффициент пропорциональности равен $\frac{\sqrt{\mu_0/\varepsilon_0}}{6\pi c^2}=\frac{20}{c^2}$:
	\begin{equation}
		P = \frac{20}{c^2}q^2a^2
	\end{equation}

	Зная мощность излучения диполя, легко найти убыль его энергии за некоторое время.

\end{document}
