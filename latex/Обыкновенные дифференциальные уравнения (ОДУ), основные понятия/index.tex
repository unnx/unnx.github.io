\documentclass{article}
\usepackage[utf8]{inputenc}
\usepackage[russian]{babel}
\usepackage{cmap}
\usepackage{amsfonts,amsmath}
\usepackage{geometry}
\usepackage{fixint}
\usepackage{rumathgrk1}
\geometry{verbose,a4paper,tmargin=1cm,bmargin=1.5cm,lmargin=0.5cm,rmargin=0.5cm}
\pdfcompresslevel=9

\begin{document}
	
	\textbf{Обыкновенные дифференциальные уравнения (ОДУ), основные понятия}\\

	ОДУ $n$-ного порядка, где $n\in\mathbb{N}$ -- это уравнение вида:
	\begin{equation}
		F(x,y,y',...y^{(n)}) = 0
	\end{equation}

	Здесь $x$ -- независимая переменная, $y=y(x)$ -- неизвестная функция, $F(p_1,p_2,...p_{n+2})$ -- заданная функция $(n+2)$-х переменных, определенная в некоторой области $D\in R^{n+2}$, причем $\frac{\partial F}{\partial p_n}\neq 0$.\\

	$n$ -- порядок ОДУ, т.е. порядок наивысшей производной $y$, входящей в $F$.\\

	Если неизвестная функция имеет вид
	\begin{equation}
		y=y(x_1,x_2,...x_m),
	\end{equation}

	То соответствующее уравнение называется дифференциальным уравнением в частных производных.\\

	Частное решение ОДУ $(1)$ -- это функция:
	\begin{equation}
		y(x) \in C^n(I),
	\end{equation}

	Где $I$ -- промежуток вида $(a,b), [a,b], [a,b), (a,b]$ ($a$ и $b$ конечные), $a<b$, $[a,+\infty]$ и т.д., такая, что при подстановке ее в $(1)$ получаем верное равенство $\forall x\in I$, т.е. выполняются два условия:
	\begin{equation}
		(x,y(x),y'(x),...y^{(n)}(x)) \in D,\;\forall x\in I
	\end{equation}
	\begin{equation}
		F(x,y(x),y'(x),...y^{(n)}(x)) = 0,\;\forall x\in I
	\end{equation}

	Общим решением ОДУ $(1)$ называется совокупность всех его частных решений.\\

	ОДУ называется разрешенным в квадратурах, если процесс его решения сводится к взятию конечного числа интегралов от известных функций.\\
\end{document}
