\documentclass{article}
\usepackage[utf8]{inputenc}
\usepackage[russian]{babel}
\usepackage{cmap}
\usepackage{amsfonts,amsmath}
\usepackage{geometry}
\usepackage{fixint}
\usepackage{rumathgrk1}
\geometry{verbose,a4paper,tmargin=1cm,bmargin=1.5cm,lmargin=0.5cm,rmargin=0.5cm}
\pdfcompresslevel=9

\begin{document}
	
	\textbf{Импульс электромагнитного поля}\\

	Когда электромагнитная волна поглащается в каком-либо теле, она оказывает на него давление. Покажем это.\\

	Пусть плоская волна падает по нормали на плоскую поверхность слабо проводящего тела с $\varepsilon$ и $\mu$ равными $1$. Электрическое поле возбудит в теле ток плотности:
	\begin{equation}
		\vec j = \sigma\vec E
	\end{equation}

	Магнитное поле волны будет действовать на единицу объема тела с силой:
	\begin{equation}
		\vec F_\text{ед. об.}=[\vec j;\vec B] = [\vec j;\vec H]
	\end{equation}

	Направление этой силы совпадает с направлением распространения волны, т.к. $\vec j || \vec E$ и $\vec E\perp\vec H$\\

	Поверхностному слою с площадью, равной единице, и толщиной $dl$ сообщается в единицу времени импульс:
	\begin{equation}
		dK = F_\text{ед. об.} dl = \mu_0 j H dl
	\end{equation}

	В этом же слое в единицу времени поглащается энергия:
	\begin{equation}
		dW = jEdl
	\end{equation}

	Она выделяется в виде теплоты.\\

	Рассмотрим отношение импульса к энергии:
	\begin{equation}
		\frac{K}{W}=\mu_0\frac{H}{E}
	\end{equation}

	Учитывая, что $\mu_0H^2 = \varepsilon E^2$, получим:
	\begin{equation}
		\frac{K}{W} = \sqrt{\varepsilon_0\mu_0}=\frac{1}{c}
	\end{equation}

	Отсюда следует, что электромагнитная волна, несущая энергию $W$, обладает импульсом:
	\begin{equation}
		K=\frac{1}{c}W
	\end{equation}

	Переходя к единице объема:
	\begin{equation}
		K_\text{ед. об.} = \frac{1}{c}w
	\end{equation}

	-- это связь между плотностью импульса и плотностью энергии.\\

	$w\vec c=\vec S$, поэтому:
	\begin{equation}
		\vec K_\text{ед. об.} = \frac{1}{c^2}\vec S = \frac{1}{c^2}[\vec E;\vec H]
	\end{equation}

	Если релятивистские частицы распределены в пространстве с плотностью $n$, плотность импульса, который они переносят, равна:
	\begin{equation}
		\vec K_\text{ед. об.} = n\frac{m\vec v}{\sqrt{1-v^2/c^2}}
	\end{equation}

	Частицы переносят с собой энергию, плотность потока которой обозначим через $\vec j_W$. Она равна плотности потока частиц, умноженной на энергию одной частицы:
	\begin{equation}
		\vec j_W = n\vec v\frac{mc^2}{\sqrt{1-v^2/c^2}}
	\end{equation}

	Поэтому:
	\begin{equation}
		\vec K_\text{ед. об.} = \frac{1}{c^2}\vec j_W
	\end{equation}

	Давление волны на единицу перпендикулярной абсолютно поглащающей поверхности равно:
	\begin{equation}
		p = <w>
	\end{equation}

	Аналогичное давление на абсолютно отражающую поверхность:
	\begin{equation}
		p = 2<w>
	\end{equation}
\end{document}
