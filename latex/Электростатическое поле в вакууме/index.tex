\documentclass{article}
\usepackage[utf8]{inputenc}
\usepackage[russian]{babel}
\usepackage{cmap}
\usepackage{amsfonts,amsmath}
\usepackage{geometry}
\usepackage{fixint}
\usepackage{rumathgrk1}
\geometry{verbose,a4paper,tmargin=1cm,bmargin=1.5cm,lmargin=0.5cm,rmargin=0.5cm}
\pdfcompresslevel=9

\begin{document}

	\textbf{Электростатическое поле в вакууме}\\
	
	Кулоном экспериментально установлено, что на точечный заряд $q_1$ в вакууме со стороны другого точеченого заряда $q_2$ действует сила:
	\begin{equation}
		\vec F_{12} = \frac{1}{4\pi\varepsilon_0}\frac{q_1q_2}{r^2}\vec e_{r}
	\end{equation}

	Вектор $\vec e_{r}$ всегда направлен в сторону того заряда, на который действует сила $\vec F$ (в данном случае от $q_2$ к $q_1$). Таким образом, если заряды имеют одинаковый знак, действие этой силы способствует их отталкиванию, и наоборот, если знаки разные, заряды притягиваются. Введя радиус-вектор $\vec r$, по модулю равный расстоянию между зарядами и направленный как $\vec e_{r}$, формулу можно переписать:
	\begin{equation}
		\vec F = \frac{1}{4\pi\varepsilon_0}\frac{q_1q_2}{r^3}\vec r
	\end{equation}

	Поскольку сила электростатического взаимодействия зависит только от зарядов и расстояния между ними, действует принцип суперпозиции:
	\begin{equation}
		\vec F_{1} = \vec F_{12} + \vec F_{13} + ... + \vec F_{1n} = \sum_{i=2}^n \vec F_{1i} 
	\end{equation}

	Если зафиксировать заряд $q$ и рассматривать силу, с которой он действует на произвольный заряд $q_1$, очевидно, что отношение этой силы к величине $q_1$ зависит только от $q$ и расстояния между зарядами. Можно ввести вспомогательную величину, показываюшую, какая сила действует со стороны зафиксированного заряда на единичный:
	\begin{equation}
		\vec E = \frac{1}{4\pi\varepsilon_0}\frac{q}{r^2}\vec e_r
	\end{equation}

	Любая группа неподвижных зарядов создает поле сил, называемое электростатическим. Величина $\vec E$ называется напряженностью такого поля. Таким образом, сила, действующая на заряд $q_1$ в поле с напряженностью $\vec E$ равна:
	\begin{equation}
		\vec F = q_1\vec E
	\end{equation}

	Чтобы соблюдался закон Кулона, вектор $\vec E$ должен быть направлен от положительных зарядов к отрицательным.\\

	Поскольку принципу суперпозиции подчиняется сила электростатического взаимодействия, этому же принципу подчиняется напряженность:
	\begin{equation}
		\vec E = \vec E_1 + ... + \vec E_n = \sum_{i=1}^n \vec E_i
	\end{equation}

	Чтобы сделать электростатическое поле наглядным, вводят линии напряженности -- ориентированные кривые, касательная к которым в любой точке совпадает по направлению с $\vec E$, и густота которых выбрана так, чтобы на единицу перпендикулярной линиям поверхности приходилось такое их количество, которое совпадало бы со значением $E$. Ясно, что изобразить все линии поля на бумаге невозможно, однако они помогают определить направление $\vec E$ и то, как этот вектор изменяется в пространстве.\\

	От векторного поля $\vec E$ можно перейти к скалярному. Требуется найти скалярную величину, определенную в каждой точке поля, показывающую, куда направлен вектор $\vec E$, и чему равен его модуль. Для этого распишем по определению работу $A$ перемещения заряда $q'$, находящегося в поле заряда $q$ из точки $1$ в точку $2$:
	\begin{equation}
		A_{12} = \int_1^2 \vec Fd\vec l = \int_1^2 F\vec e_r d\vec l
	\end{equation}

	Скалярное произведение $(\vec e_r; d\vec l)$ дает приращение модуля радиус-вектора $dr$. Таким образом, криволинейный интеграл сводится к определенному:
	\begin{equation}
		A_{12} = \int_{r_1}^{r_2} Fdr = \frac{qq'}{4\pi\varepsilon_0}\int_{r_1}^{r_2}\frac{dr}{r^2} = \frac{1}{4\pi\varepsilon_0}(\frac{qq'}{r_1}-\frac{qq'}{r_2})
	\end{equation}

	Поскольку сила электростатического взаимодействия центральна, ее поле консервативно. Следовательно, работа сил поля представима как убыль потенциальной энергии заряда:
	\begin{equation}
		A_{12} = W_{p1} - W_{p2}
	\end{equation}

	Отсюда:
	\begin{equation}
		W_{p} = \frac{1}{4\pi\varepsilon_0}\frac{qq'}{r} + const
	\end{equation}

	Константа $const$ выбирается таким образом, чтобы потенциальная энергия $q'$ на бесконечном удалении от заряда, создающего поле, была равной нулю. Ясно, что $const=0$.\\

	Поле создано зарядом $q$, заряд $q'$ может быть произвольным. Отношение $\frac{W_p}{q'}$ зависит только от поля. Мы нашли некую скалярную величину, характеризующую поле в каждой его точке:
	\begin{equation}
		\varphi = \frac{W_p}{q'} = \frac{1}{4\pi\varepsilon_0}\frac{q}{r}
	\end{equation}

	Эта величина называется потенциалом электростатического поля.\\

	Аддитивность потенциальной энергии и принцип суперпозиции $\vec E$ дают аддитивность потенциала:
	\begin{equation}
		\varphi = \sum_{i=1}^n \varphi_i = \frac{1}{4\pi\varepsilon_0}\sum_{i=1}^{n}\frac{q_i}{r_i}
	\end{equation}

	Таким образом, потенциальная энергия заряда равна:
	\begin{equation}
		W_p = q'\varphi
	\end{equation}

	Похоже на случай с напряженностью.\\

	Из консервативности сил поля $\vec E$, во-первых, следует, что работа этих сил по замкнутому контуру равна нулю ($r_1=r_2$). Во-вторых, соблюдаются следующие соотношения:
	\begin{equation}
		E_x = -\frac{\partial\varphi}{\partial x}; \;\;\;
		E_y = -\frac{\partial\varphi}{\partial y}; \;\;\;
		E_z = -\frac{\partial\varphi}{\partial z}
	\end{equation}

	Это можно переписать с помощью дифференциального оператора набла:
	\begin{equation}
		\vec E = -(\frac{\partial\varphi}{\partial x}\vec e_x + \frac{\partial\varphi}{\partial y}\vec e_y + \frac{\partial\varphi}{\partial z}\vec e_z) = -\vec\nabla \varphi
	\end{equation}

	Получили связь напряженности и потенциала, нашли скалярную величину, характеризующую изменение напряженности поля в пространстве.\\

	Для наглядности важно понимать, что напряженность в любой точке направлена в сторону убыли потенциала.\\

	С помощью соотношения $\vec E = -\vec\nabla\varphi$ можно найти напряженность, зная потенциал. Обратная операция тоже осуществима и описана нами выше:
	\begin{equation}
		\varphi_2-\varphi_1 = \int_1^2 \vec E d\vec l
	\end{equation}

\end{document}
