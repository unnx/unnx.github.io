\documentclass{article}
\usepackage[utf8]{inputenc}
\usepackage[russian]{babel}
\usepackage{cmap}
\usepackage{amsfonts,amsmath}
\usepackage{geometry}
\usepackage{fixint}
\usepackage{rumathgrk1}
\geometry{verbose,a4paper,tmargin=1cm,bmargin=1.5cm,lmargin=0.5cm,rmargin=0.5cm}
\pdfcompresslevel=9

\begin{document}
	\textbf{Лемма о частном решении ОДУ 1-го порядка со специальной правой частью}\\

	Пусть дано уравнение:
	\begin{equation}
		y' - \lambda y = P_m(x)e^{\mu x}
	\end{equation}

	Тогда по лемме:\\

	При $\lambda\neq\mu$ (нерезонансный случай) ОДУ всегда имеет частное решение вида:
	\begin{equation}
		y_\text{ч.н.} = Q_m(x)e^{\mu x}
	\end{equation}

	При $\lambda = \mu$ (резонансный случай):
	\begin{equation}
		y_\text{ч.н.} = xQ_m(x)e^{\mu x}
	\end{equation}

	Доказательство:\\

	Ищем частное решение в виде:
	\begin{equation}
		y=ze^{\mu x},\;\;z - ?
	\end{equation}

	Отсюда:
	\begin{equation}
		y'=z'e^{\mu x} + z\mu e^{\mu x}
	\end{equation}

	Подставляем в $(1)$:
	\begin{equation}
		z'e^{\mu x} + z\mu e^{\mu x} - \lambda ze^{\mu x} = P_m(x)e^{\mu x}
	\end{equation}

	Делим на $e^{\mu x}$:
	\begin{equation}
		z' + z(\mu - \lambda) = P_m(x)
	\end{equation}

	1) Пусть $\lambda\neq\mu$. Будем искать частное решение уравнения $(7)$ в виде многочлена:
	\begin{equation}
		z=Q_m(x)=b_0x^m+b_1x^{m-1}+...+b_m
	\end{equation}

	Полагаем:
	\begin{equation}
		P_m(x)=a_0x^m+a_1x^{m-1}+...+a_m
	\end{equation}

	Подставляем все в $(1)$:
	\begin{equation}
		mb_0x^{m-1}+(m-1)b_1x^{m-2}+...+b_{m-1}+(\mu-\lambda)(b_0x^m+b_1x^{m-1}+...+b_m)=a_0x^m+a_1x^{m-1}+...+a_m
	\end{equation}

	Методом неопределенных коэффициентов находим $b_0..b_m$:
	\begin{equation}
		x^m: (m-\lambda)b_0 = a_0 \Rightarrow b_0=\frac{a_0}{m-\lambda}\neq 0
	\end{equation}
	\begin{equation}
		x^{m-1}: mb_0 = a_1 \Rightarrow b_1=\frac{a_1-mb_0}{m-\lambda}
	\end{equation}

	и т.д.\\

	Записываем частное решение:
	\begin{equation}
		z_\text{ч.н.}=Q_m(x)\Rightarrow y_\text{ч.н.} = Q_m(x)e^{\mu x}
	\end{equation}

	2) Пусть $\lambda=\mu$. Тогда уравнение принимает вид:
	\begin{equation}
		z'=P_m(x)
	\end{equation}

	Частное решение:
	\begin{equation}
		z_\text{ч.н.} = \frac{a_0}{m+1}x^{m+1}+\frac{a_1}{m}x^m+...+a_mx=x(\frac{a_0}{m+1}x^m+\frac{a_1}{m}x^{m-1}+...+a_m)=xQ_m(x)
	\end{equation}

	Доказано.\\

	Принцип суперпозиции:\\

	Если уравнение представлено в виде:
	\begin{equation}
		l(D)[y] = \sum_{j=1}^N f_j(x)
	\end{equation}

	И $y_{\text{ч.н.}_j}$ -- частное решение уравнения вида $l(D)[y] = f_j(x)$, то частное решение исходного уравнения -- есть сумма соответсвующих частных решений $y_{\text{ч.н.}_j}$:
	\begin{equation}
		y_{\text{ч.н.}} = \sum_{j=1}^N y_{\text{ч.н.}_j}
	\end{equation}

\end{document}
