\documentclass{article}
\usepackage[utf8]{inputenc}
\usepackage[russian]{babel}
\usepackage{cmap}
\usepackage{amsfonts,amsmath}
\usepackage{geometry}
\usepackage{fixint}
\usepackage{rumathgrk1}
\geometry{verbose,a4paper,tmargin=1cm,bmargin=1.5cm,lmargin=0.5cm,rmargin=0.5cm}
\pdfcompresslevel=9

\begin{document}
	
	\textbf{Уравнения Максвелла}\\

	Открытие тока смещения позволило Максвеллу создать единую теорию электрических и магнитных явлений. Она полностью покрывается уравнениями Максвелла.\\

	Первая пара:
	\begin{equation}
		[\vec\nabla;\vec E] = -\frac{\partial B}{\partial t}
	\end{equation}
	\begin{equation}
		(\vec\nabla;\vec B) = 0
	\end{equation}

	Вторая пара:
	\begin{equation}
		[\vec\nabla;\vec H] = \vec j + \frac{\partial\vec D}{\partial t}
	\end{equation}
	\begin{equation}
		(\vec\nabla;\vec D) = \rho
	\end{equation}

	Пользуясь теоремами Остроградского-Гаусса и Стокса, можно переписать уравнения Максвелла в интегральной форме.\\

	Первая пара в интегральной форме:
	\begin{equation}
		\oint_l \vec E d\vec l = -\frac{\partial}{\partial t}\int_S \vec B d\vec S
	\end{equation}
	\begin{equation}
		\oint_S \vec B d\vec S = 0
	\end{equation}

	Вторая пара в интегральной форме:
	\begin{equation}
		\oint_l \vec H d\vec l = \int_S \vec j d\vec S + \frac{\partial}{\partial t} \int_S \vec D d\vec S
	\end{equation}
	\begin{equation}
		\oint_l \vec D d\vec S = \int_V \rho dV
	\end{equation}


\end{document}
