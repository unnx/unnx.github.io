\documentclass{article}
\usepackage[utf8]{inputenc}
\usepackage[russian]{babel}
\usepackage{cmap}
\usepackage{amsfonts,amsmath}
\usepackage{geometry}
\usepackage{fixint}
\usepackage{rumathgrk1}
\geometry{verbose,a4paper,tmargin=1cm,bmargin=1.5cm,lmargin=0.5cm,rmargin=0.5cm}
\pdfcompresslevel=9

\begin{document}
	
	\textbf{Проводник в электрическом поле}\\

	Проводником является вещество с большим количеством свободных носителей заряда. При малейшей силе, действующей на эти носители, внутри проводника возникает электрический ток.\\

	Отсюда следует, что для равновесия зарядов в проводнике необходимо выполнение двух условий: 1) внутри проводника $\vec E = 0$ и 2) напряженность на поверхности проводника направлена по нормали к поверхности. \\

	Из этих двух условий следует, что при сообщении проводнику заряда $q$, он распределяется по поверхности. Аналогичный вывод можно сделать, учитывая, что заряды одного знака отталкиваются и стремятся распределиться с наименьшей объемной плотностью. Отсюда, в частности, следует тот факт, что заряды распределяются по поверхности проводника с постоянной поверхностной плотностью $\sigma$.\\

	Потенциал проводника совпадает с потенциалом его поверхности. Требуется найти величину, связывающую потенциал и заряд на проводнике. Такой коэффициент пропорциональности называется электроемкостью проводника $C$:
	\begin{equation}
		q = C\varphi
	\end{equation}

	$\varphi$ равен работе по перемещению единичного заряда из бесконечности на поверхность проводника, а электроемкость численно равна заряду, увеличивающему потенциал проводника на единицу.\\

	Заряд, распределенный по поверхности проводника, можно представить как систему зарядов. Ее потенциальная энергия, как известно, равна:
	\begin{equation}
		W_p = \frac{1}{2}\sum_{i}^n q_i\varphi_i
	\end{equation}

	Но поверхность проводника эквипотенциальна, поэтому энергия представима в виде:
	\begin{equation}
		W_p = \frac{q\varphi}{2}
	\end{equation}

	Раскрыв потенциал через электроемкость проводника, получим:
	\begin{equation}
		W_p = \frac{q\varphi}{2} = \frac{q^2}{2C}
	\end{equation}

	Аналогично можно раскрыть и заряд:
	\begin{equation}
		W_p = \frac{C\varphi^2}{2}
	\end{equation}

\end{document}
