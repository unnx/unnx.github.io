\documentclass{article}
\usepackage[utf8]{inputenc}
\usepackage[russian]{babel}
\usepackage{cmap}
\usepackage{amsfonts,amsmath}
\usepackage{geometry}
\usepackage{fixint}
\usepackage{rumathgrk1}
\geometry{verbose,a4paper,tmargin=1cm,bmargin=1.5cm,lmargin=0.5cm,rmargin=0.5cm}
\pdfcompresslevel=9

\begin{document}
	
	\textbf{Уравнения Бернулли и Риккати}\\

	Уравнением Бернулли называется уравнение
	\begin{equation}
		y' + a(x)y = b(x)y^\alpha
	\end{equation}

	При $\alpha\neq 0$ и $\alpha\neq 1$ (легко видеть, что в противном случае получаются линейные ОДУ 1-го порядка, неоднородное и однородное соответственно).\\

	Метод решения: разделить на $y^\alpha$:
	\begin{equation}
		\frac{y'}{y^\alpha} + a(x)y^{1-\alpha} = b(x)
	\end{equation}

	Сделать замену $z = y^{1-\alpha}$ (тогда $z'=(1-\alpha)y^{-\alpha}y'$, т.е. $\frac{y'}{y^\alpha}=\frac{z'}{1-\alpha}$):
	\begin{equation}
		\frac{1}{1-\alpha}z' + a(x)z = b(x)
	\end{equation}

	Решить полученное уравнение и в общем решении сделать обратную подстановку.\\

	Уравнение Риккати:
	\begin{equation}
		y' + a(x)y + b(x)y^2 = c(x)
	\end{equation}

	В общем случае не решается в квадратурах. Если известно одно частное решение $y_1(x)$, то заменой $y = y_1(x)+z$ уравнение Риккати сводится к уравнению Бернулли.\\

	Свойства.\\

	1) В уравнении Риккати можно провести замену:
	\begin{equation}
		x = \varphi(\widetilde x)
	\end{equation}

	Где $\widetilde x$ -- гладкая функция, либо дробно-линейную замену:
	\begin{equation}
		y = \frac{\alpha(x)\widetilde y + \beta x}{\gamma(x)\widetilde y + \delta x},\;\; \begin{vmatrix} \alpha & \beta \\ \gamma & \delta \end{vmatrix}\neq 0
	\end{equation}

	2) Заменой $y = z\varphi(x)$, $z = w + \psi(x)$ уравнение Риккати можно привести к виду:
	\begin{equation}
		w' = w + f(x)
	\end{equation}

	3) Уравнение Риккати можно привести к линейному однородному ОДУ 2-го порядка (с переменными коэффициентами):
	\begin{equation}
		y = a(x)z,\;\;z = -\frac{u'}{u}
	\end{equation}

	Частный случай уравнения Риккати:
	\begin{equation}
		y' + y^2 = \frac{a}{x^2}
	\end{equation}

	Лиувилль доказал, что это уравнение решается только при $\alpha=2$ и $\alpha = \frac{4k}{2k-1},\;k\in\mathbb{Z}$.

\end{document}
