\documentclass{article}
\usepackage[utf8]{inputenc}
\usepackage[russian]{babel}
\usepackage{cmap}
\usepackage{amsfonts,amsmath}
\usepackage{geometry}
\usepackage{fixint}
\usepackage{rumathgrk1}
\geometry{verbose,a4paper,tmargin=1cm,bmargin=1.5cm,lmargin=0.5cm,rmargin=0.5cm}
\pdfcompresslevel=9

\begin{document}
	
	\textbf{Конденсаторы}\\

	Конденсатором называется система из двух проводников, находящихся на небольшом расстоянии друг от друга и разделенных диэлектриком.\\

	Основная характеристика конденсатора -- электроемкость $C$:
	\begin{equation}
		q = C(\varphi_1-\varphi_2)
	\end{equation}

	Здесь $q$ -- заряд конденсатора, а $\varphi_1, \varphi_2$ -- потенциалы двух проводников.\\

	Разность потенциалов $\varphi_1 - \varphi_2$ называется напряжением $U$ между обкладками конденсатора (обычно проводники выбираются плоские, сферические, и т.д., поэтому и называются обкладками).\\

	Найдем емкость плоского конденсатора. Напряженность между обкладками примерно равна напряженности между двумя разноименно заряженными плоскостями.
	\begin{equation}
		E = \frac{\sigma}{\varepsilon_0\varepsilon} = \frac{q}{\varepsilon_0\varepsilon S}
	\end{equation}

	Коэффициент $\frac{1}{\varepsilon}$ появился из того соображения, что пространство между плоскостями заполнено диэлектриком с проницаемостью $\varepsilon$.\\

	Плоскости являются эквипотенциальными, поэтому:
	\begin{equation}
		\varphi_1-\varphi_2 = Ed = \frac{qd}{\varepsilon_0\varepsilon S}
	\end{equation}

	Отсюда:
	\begin{equation}
		C = \frac{\varepsilon_0\varepsilon S}{d}
	\end{equation}

	Энергию заряженного конденсатора легко найти как энергию системы зарядов:
	\begin{equation}
		W_p = \frac{1}{2}q(\varphi_1-\varphi_2) = \frac{1}{2}qU
	\end{equation}

	Отсюда:
	\begin{equation}
		W_p = \frac{qU}{2} = \frac{q^2}{2C} = \frac{CU^2}{2}
	\end{equation}

\end{document}
