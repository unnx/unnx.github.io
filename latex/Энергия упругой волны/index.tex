\documentclass{article}
\usepackage[utf8]{inputenc}
\usepackage[russian]{babel}
\usepackage{cmap}
\usepackage{amsfonts,amsmath}
\usepackage{geometry}
\usepackage{fixint}
\usepackage{rumathgrk1}
\geometry{verbose,a4paper,tmargin=1cm,bmargin=1.5cm,lmargin=0.5cm,rmargin=0.5cm}
\pdfcompresslevel=9

\begin{document}
	
	\textbf{Энергия упругой волны}\\

	Уравнение упругой волны:
	\begin{equation}
		\xi = a\cos(\omega t - kx + \varphi)
	\end{equation}

	Рассмотрим элементарный объем $\Delta V$ такой, что
	\begin{equation}
		\frac{\partial\xi}{\partial t} = const;\;\;\frac{\partial\xi}{\partial x} = const
	\end{equation}

	Выведем кинетическую энергию выделенного объема:
	\begin{equation}
		\Delta W_\text{к} = \frac{mv^2}{2} = \frac{\rho\Delta V}{2}(\frac{\partial\xi}{\partial t})^2
	\end{equation}

	Теперь выведем потенциальную энергию:
	\begin{equation}
		\Delta W_\text{п} = \frac{k\Delta l^2}{2} = \{\varepsilon=\frac{\Delta l}{L},\;k=\frac{ES}{L}\} = \frac{ES}{L}\frac{(\varepsilon L)^2}{2}=\frac{ESL}{2}\varepsilon^2=\frac{ESL}{2}(\frac{\partial\xi}{\partial x})^2=\frac{E}{2}(\frac{\partial\xi}{\partial x})^2\Delta V
	\end{equation}

	Отсюда потенциальная энергия данного объема:
	\begin{equation}
		\Delta W_\text{п}=\frac{\rho v^2}{2}(\frac{\partial\xi}{\partial x})^2 \Delta V
	\end{equation}

	Найдем полную энергию:
	\begin{equation}
		\Delta W = \Delta W_\text{к}+\Delta W_\text{п} = \frac{1}{2}\rho[(\frac{\partial\xi}{\partial t})^2+v^2(\frac{\partial\xi}{\partial x})^2]\Delta V
	\end{equation}

	Получим плотность энергии $w=\frac{\Delta W}{\Delta V}$:
	\begin{equation}
		w=\frac{1}{2}\rho[(\frac{\partial\xi}{\partial t})^2+v^2(\frac{\partial\xi}{\partial x})^2]
	\end{equation}

	Подставим в выражение для плотности энергии производные $\xi$:
	\begin{equation}
		\frac{\partial\xi}{\partial t} = ka\sin(\omega t - kx + \varphi)
	\end{equation}
	\begin{equation}
		\frac{\partial\xi}{\partial x} = -a\omega\sin(\omega t - kx + \varphi)
	\end{equation}

	Получится это:
	\begin{equation}
		w = \rho a^2\omega^2\sin^2(\omega t - kx + \varphi)
	\end{equation}

	Поскольку $<\sin^2 t>=\frac{1}{2}$, среднее по времени значение плотности энергии равно:
	\begin{equation}
		<w> = \frac{1}{2}\rho a^2\omega
	\end{equation} 

	Потоком энергии называется производная от энергии по времени:
	\begin{equation}
		\Phi = \frac{dW}{dt}
	\end{equation}

	Рассматривают также плотность потока энергии. Она равна отношению величины потока через перпендикулярную ему поверхность к площади этой поверхности (обозначается $j$):
	\begin{equation}
		\frac{\Delta\Phi}{\Delta S_\perp} = \frac{\Delta W}{\Delta S_\perp\Delta t} = j
	\end{equation}

	Выразим $j$ через фазовую скорость ($\Delta V$ -- объем косого цилиндра, длину высоты которого волна, двигаясь с фазовой скоростью $\Delta v$ проходит за $\Delta t$):
	\begin{equation}
		\Delta W = w\Delta V = w\Delta S_\perp v\Delta t
	\end{equation}

	Отсюда:
	\begin{equation}
		j = \frac{\Delta W}{\Delta S_\perp\Delta t} = wv
	\end{equation}

	Скорость -- вектор, поэтому $j$ тоже может быть вектором:
	\begin{equation}
		\vec j = w\vec v
	\end{equation}

	Он называется вектором Умова.\\

	Через среднее значение плотности энергии можно выразить среднее значение вектора Умова:
	\begin{equation}
		<\vec j> = <w>v = \frac{1}{2}\rho a^2 \omega^2 \vec v
	\end{equation}

	Поток вектора Умова через поверхность есть поток энергии через нее:
	\begin{equation}
		\Phi = \int_S\vec j d\vec S
	\end{equation}

	Аналогично для среднего потока энергии:
	\begin{equation}
		<\Phi> = \int_S \vec jd\vec S
	\end{equation}

	Рассмотрим специально случай сферической волны. Вектор Умова для нее всюду перпендикулярен волновой поверхности. Найдем средний поток энергии:
	\begin{equation}
		<\Phi>=\int_S <j>dS_n = <j>S = <j>4\pi r^2= 2\pi\rho\omega^2 v a_r^2 r^2
	\end{equation}

	Здесь $a_r$ -- амплитуда на расстоянии $r$ от источника. Если энергия не поглащается средой, то $<\Phi>=const$, поэтому:
	\begin{equation}
		a_r^2r^2 = const,\;\;a_r\sim\frac{1}{r}
	\end{equation}

	Если среда поглащает энергию, присутствует затухание вектора Умова:
	\begin{equation}
		<j> = j_0e^{-kx},\;\;k=2\gamma
	\end{equation}

	($\gamma$ -- коэффициент поглащения)


\end{document}
