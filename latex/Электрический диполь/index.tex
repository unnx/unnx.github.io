\documentclass{article}
\usepackage[utf8]{inputenc}
\usepackage[russian]{babel}
\usepackage{cmap}
\usepackage{amsfonts,amsmath}
\usepackage{geometry}
\usepackage{fixint}
\usepackage{rumathgrk1}
\geometry{verbose,a4paper,tmargin=1cm,bmargin=1.5cm,lmargin=0.5cm,rmargin=0.5cm}
\pdfcompresslevel=9

\begin{document}

	\textbf{Электрический диполь}\\

	Электрическим диполем называется система из двух одинаковых по модулю и различных по знаку зарядов $+q$ и $-q$, разнесенных на постоянное расстояние $l$.

	Отрезок, соединяющий заряды, называется осью диполя. Эту ось принято обозначать через вектор $\vec l$, равный по модулю расстоянию между зарядами и направленный от $-q$ к $+q$.

	Электрическим дипольным моментом называется произведение модуля одного из зарядов на вектор $\vec l$:
	\begin{equation}
		\vec p = |q|\vec l
	\end{equation}

	Центром диполя считается центр его оси. Введя полярную систему координат, в которой полярная ось совпадает по направлению с $l$, а центр находится в центре диполя, любую точку пространства относительно диполя можно представить в виде сочетания координат $(r,\vartheta)$, либо как вектор $\vec r$.\\

	Найдем потенциал поля диполя. Он складывается из потенциала полей первого и второго зарядов:
	\begin{equation}
		\varphi = \frac{1}{4\pi\varepsilon_0}(\frac{+q}{r_+}+\frac{-q}{r_-})=\frac{|q|}{4\pi\varepsilon_0}(\frac{1}{r_+}-\frac{1}{r_-})
	\end{equation}

	Здесь $r_+$ и $r_-$ -- расстояния до $+q$ и $-q$ соответственно. Таким образом:
	\begin{equation}
		\varphi = \frac{|q|}{4\pi\varepsilon_0}\frac{r_--r_+}{r_+r_-}
	\end{equation}

	Введем векторы $+\vec a$ и $-\vec a$, выходящие из центра диполя, по модулю равные половине длины его оси и направленные в стороны положительного и отрицательного зарядов соответственно. Зная вектор $\vec r$, проведенный из центра диполя в произвольную точку пространства, можем приближенно найти расстояния от этой точки до зарядов:
	\begin{equation}
		r_+ = r - a\cos\vartheta = r - (\vec a;\vec e_r)
	\end{equation}
	\begin{equation}
		r_- = r + a\cos\vartheta = r + (\vec a;\vec e_r)
	\end{equation}

	Вернемся к формуле $(3)$. Произведение $r_+r_-$ с высокой точностью можно представить как $r^2$. Разность $r_--r_+$ раскрывается по формулам $(4)$ и $(5)$:
	\begin{equation}
		r_--r_+=2a\cos\vartheta=(2\vec a;\vec e_r)=(\vec l;\vec e_r)
	\end{equation}

	Отсюда:
	\begin{equation}
		\varphi = \frac{1}{4\pi\varepsilon_0}\frac{(|q|\vec l;\vec e_r)}{r^2} = \frac{1}{4\pi\varepsilon_0}\frac{(\vec p;\vec e_r)}{r^2}
	\end{equation}

	Можно раскрыть скалярное произведение:
	\begin{equation}
		\varphi = \frac{1}{4\pi\varepsilon_0}\frac{p\cos\vartheta}{r^2}
	\end{equation}

	Чтобы найти напряженность поля диполя, найдем ее проекции на вектор, перпендикулярный радиус-вектору и на сам радиус-вектор ($E_\vartheta$ и $E_r$ соответственно):
	\begin{equation}
		E_r = -\frac{d\varphi}{dr} = \frac{1}{4\pi\varepsilon_0}\frac{2p\cos\vartheta}{r^3}
	\end{equation}
	\begin{equation}
		E_\vartheta = -\frac{d\varphi}{rd\vartheta} = \frac{1}{4\pi\varepsilon_0}\frac{p\sin\vartheta}{r^3}
	\end{equation}

	Квадрат напряженности равен сумме квадратов найденых проекций (т.к. оси, на которые они сделаны, образуют прямоугольную декартову систему координат):
	\begin{equation}
		E^2 = (\frac{p}{4\pi r^3\varepsilon_0})^2(4\cos^2\vartheta+\sin^2\vartheta) = (\frac{p}{4\pi r^3\varepsilon_0})^2(1+3\cos^2\vartheta)
	\end{equation}

	Найдем модуль напряженностив:
	\begin{equation}
		E = \frac{1}{4\pi\varepsilon_0}\frac{p}{r^3}\sqrt{1+3\cos^2\vartheta} 
	\end{equation}

	Если внести диполь в поле $\vec E$, на каждый заряд будет действовать сила:
	\begin{equation}
		\vec F = q\vec E
	\end{equation}

	Модуль момента пары сил, действующих на диполь, равен произведению модуля любой силы на плечо:
	\begin{equation}
		N = |q|El\sin\vartheta = pE\sin\vartheta
	\end{equation}

	Можно переписать это соотношение в векторном виде:
	\begin{equation}
		\vec N = [\vec p;\vec E]
	\end{equation}

	Потенциальная энергия диполя относительно поля, в которое он помещен, равна:
	\begin{equation}
		W_p = |q|(\varphi_+ - \varphi_-)
	\end{equation}

	Здесь $\varphi_+$ и $\varphi_-$ -- потенциалы внешнего поля в тех точках, где находятся заряды $q_+$ и $q_-$. Введя ось $x$, совпадающую по направлению с $\vec E$, распишем выражение для потенциальной энергии:
	\begin{equation}
		W_p = |q|(\varphi_+ - \varphi_-) = |q|\frac{\partial\varphi}{\partial x}\Delta x
	\end{equation}

	$\Delta x$ -- модуль проекции оси диполя на ось $x$. Отсюда:
	\begin{equation}
		W_p = |q|El\cos\vartheta = pE\cos\vartheta
	\end{equation}

	Либо в векторном виде:
	\begin{equation}
		W_p = (\vec p;\vec E)
	\end{equation}






\end{document}
	