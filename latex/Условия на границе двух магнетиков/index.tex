\documentclass{article}
\usepackage[utf8]{inputenc}
\usepackage[russian]{babel}
\usepackage{cmap}
\usepackage{amsfonts,amsmath}
\usepackage{geometry}
\usepackage{fixint}
\usepackage{rumathgrk1}
\geometry{verbose,a4paper,tmargin=1cm,bmargin=1.5cm,lmargin=0.5cm,rmargin=0.5cm}
\pdfcompresslevel=9

\begin{document}
	
	\textbf{Условия на границе двух магнетиков}\\

	Выясним, как изменяются $\vec B$ и $\vec H$ при переходе из одного магнетика в другой. Будем руководствоваться известными соотношениями:
	\begin{equation}
		(\vec\nabla;\vec B) = 0; \;\;\; [\vec\nabla;\vec H] = \vec j
	\end{equation}

	Выберем поверхность параллелепипеда, пересекающую границу раздела. Площади оснований -- $S$, высота $h\rightarrow 0$. Выберем также нормаль $\vec n$. Поток вектора $\vec B$ через эту замкнутую поверхность будет равен: 
	\begin{equation}
		B_{1_n}S - B_{2_n}S = 0
	\end{equation}

	Отсюда:
	\begin{equation}
		B_{1n} = B_{2n}
	\end{equation}

	Теперь выберем прямоугольный контур, охватывающий часть границы раздела, со сторонами $a$ и $h\rightarrow 0$. Выберем также направление тангенциали $\vec\tau$. По теореме Стокса:
	\begin{equation}
		\oint \vec H d\vec l = H_{1\tau}a - H_{2\tau}a = 0
	\end{equation}

	(Поток равен $0$, т.к. по границе раздела не текут макроскопические токи)\\

	Получаем:
	\begin{equation}
		H_{1\tau} = H_{2\tau}
	\end{equation}

	Легко найти изменения составляющих $H_n$ и $B_\tau$. Выразим $B$ через $H$:
	\begin{equation}
		\mu_0\mu_1 H_{1n} = \mu_0\mu_2 H_{2n}
	\end{equation}
	\begin{equation}
		\frac{H_{1n}}{H_{2n}} = \frac{\mu_2}{\mu_1}
	\end{equation}

	Теперь сделаем обратное:
	\begin{equation}
		\frac{B_{1\tau}}{\mu_0\mu_1} = \frac{B_{2\tau}}{\mu_0\mu_2}
	\end{equation}
	\begin{equation}
		\frac{B_{1\tau}}{B_{2\tau}} = \frac{\mu_1}{\mu_2}
	\end{equation}

	Получили группу соотношений:
	\begin{equation}
		B_{1n} = B_{2n}, \;\;\; \frac{B_{1\tau}}{B_{2\tau}} = \frac{\mu_1}{\mu_2}
	\end{equation}
	\begin{equation}
		H_{1\tau} = H_{2\tau}, \;\;\; \frac{H_{1n}}{H_{2n}} = \frac{\mu_2}{\mu_1}
	\end{equation}

\end{document}
