\documentclass{article}
\usepackage[utf8]{inputenc}
\usepackage[russian]{babel}
\usepackage{cmap}
\usepackage{amsfonts,amsmath}
\usepackage{geometry}
\usepackage{fixint}
\usepackage{rumathgrk1}
\geometry{verbose,a4paper,tmargin=1cm,bmargin=1.5cm,lmargin=0.5cm,rmargin=0.5cm}
\pdfcompresslevel=9

\begin{document}
	
	\textbf{Магнитное поле в вакууме}\\

	Экспериментально установлено, что движущиеся заряды действуют с некоторой силой на другие движущиеся заряды, а токи действуют на токи.\\

	Из опыта следует, что два параллельных провода с током одного направления притягиваются с силой:
	\begin{equation}
		F = k\frac{2I_1I_2}{b}
	\end{equation}

	Здесь $b$ -- расстояние между проводами. \\

	Вводится поле этих сил, называемое магнитным.\\

	Введем для замкнутого проводника (контура) площади $S$, по которому течет ток $I$, специальную характеристику, называемую магнитным моментом контура:
	\begin{equation}
		\vec p_m = I\vec S
	\end{equation} 

	На контур с током в магнитном поле со стороны этого поля действуют силы, создающие вращающий момент. Этот момент зависит от положения контура в пространстве и $\vec p_m$, но соотношение $\frac{M_{max}}{p_m}$ зависит только от магнитного поля. Вводится характеристика поля, называемая индукцией $\vec B$:
	\begin{equation}
		B = \frac{M_{max}}{p_m}
	\end{equation}

	(Вектор $\vec B$ направлен по нормали к контуру)\\

	Для индукции магнитного поля верен принцип суперпозиции:
	\begin{equation}
		\vec B = \sum_i \vec B_i
	\end{equation}

	Био, Савар и Лаплас вывели закон, выражающий элементарную индукцию:
	\begin{equation}
		d\vec B = \frac{\mu_0}{4\pi}\frac{I[d\vec l;\vec r]}{r^3}
	\end{equation}

	(Вектор $d\vec l$ сонаправлен с вектором плотности тока)\\

	Отсюда легко вывести магнитное поле движущегося заряда:
	\begin{equation}
		I = jS = ne'vS
	\end{equation}
	\begin{equation}
		d\vec B = \frac{\mu_0}{4\pi}\frac{ne'vS[d\vec l;\vec r]}{r^3}
	\end{equation}
	\begin{equation}
		d\vec B = \frac{\mu_0}{4\pi}\frac{ne'Sdl[\vec v;\vec r]}{r^3}
	\end{equation}
	\begin{equation}
		d\vec B = \frac{\mu_0}{4\pi}\frac{Ne'[\vec v;\vec r]}{r^3}
	\end{equation}
	\begin{equation}
		d\vec B = \frac{\mu_0}{4\pi}\frac{q[\vec v;\vec r]}{r^3}
	\end{equation}

\end{document}
