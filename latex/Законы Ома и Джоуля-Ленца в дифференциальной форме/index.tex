\documentclass{article}
\usepackage[utf8]{inputenc}
\usepackage[russian]{babel}
\usepackage{cmap}
\usepackage{amsfonts,amsmath}
\usepackage{geometry}
\usepackage{fixint}
\usepackage{rumathgrk1}
\geometry{verbose,a4paper,tmargin=1cm,bmargin=1.5cm,lmargin=0.5cm,rmargin=0.5cm}
\pdfcompresslevel=9

\begin{document}
	
	\textbf{Законы Ома и Джоуля-Ленца в дифференциальной форме}\\

	Закон Ома для однородного участка цепи гласит:
	\begin{equation}
		I = \frac{U}{R}
	\end{equation}

	При этом
	\begin{equation}
		R = \rho\frac{l}{S}
	\end{equation}

	Перейдем к плотности тока в цилиндре длины $dl$ с основанием $dS$:
	\begin{equation}
		jdS = \frac{UdS}{\rho dl} = \frac{dS}{\rho dl}Edl
	\end{equation}
	\begin{equation}
		j = \frac{1}{\rho} E
	\end{equation}

	Введя величину $\sigma = \frac{1}{\rho}$, называемую удельной электрической проводимостью, получаем:
	\begin{equation}
		\vec j = \sigma\vec E
	\end{equation}

	Полученное выражение -- закон Ома для однородного участка цепи в дифференциальной форме.\\

	В случае неоднородного участка цепи имеется поле сторонних сил $\vec E^*$:
	\begin{equation}
		\vec j = \sigma(\vec E + \vec E^*)
	\end{equation}

	Закон Джоуля-Ленца в интегральной форме утверждает:
	\begin{equation}
		Q = UIt = RI^2 t
	\end{equation}

	Перейдем к элементарным величинам:
	\begin{equation}
		dQ = RI^2 dt
	\end{equation}

	Раскроем сопротивление:
	\begin{equation}
		dQ = \frac{\rho dl}{dS}I^2 dt
	\end{equation}

	И силу тока:
	\begin{equation}
		dQ = \frac{\rho dl}{dS}(j dS)^2 dt
	\end{equation}

	Получаем:
	\begin{equation}
		dQ = \rho j^2 dV dt
	\end{equation}

	Разделив на $dVdt$ получаем количество теплоты, выделяемое в единице объема за единицу времени $Q_\text{уд}$ (поскольку $j$ стоит в квадрате, можно перейти к вектору $\vec j $):
	\begin{equation}
		Q_\text{уд} = \rho \vec j\;^2
	\end{equation}

	Полученная формула выражает закон Джоуля-Ленца в дифференциальной форме. Она определяет удельную мощность тока:
	\begin{equation}
		P_\text{уд} = \rho \vec j\;^2
	\end{equation}

	Аналогичный результат будет, если в формулу для удельной мощности $P_\text{уд} = (\vec j;\vec E + \vec E^*)$ подставить $\vec E + \vec E^* = \rho \vec j$.
\end{document}
